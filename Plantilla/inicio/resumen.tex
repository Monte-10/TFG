La intersección entre deporte y nutrición es un tema ampliamente discutido debido a su importancia en el mantenimiento de un estilo de vida saludable. A pesar de la abundancia de información disponible, muchas personas enfrentan desafíos para integrar prácticas saludables en su vida cotidiana debido a la falta de tiempo y conocimientos específicos. Este dilema se traduce en la adopción de dietas genéricas o rutinas de ejercicio inadecuadas, aumentando el riesgo de resultados negativos y lesiones. La barrera principal para adoptar un estilo de vida saludable radica en la insuficiencia de información relevante accesible y en la limitación de tiempo para adquirir dichos conocimientos.

Reconociendo esta necesidad, mi Trabajo Final de Grado (TFG) se centra en simplificar el acceso a planes de nutrición y ejercicio personalizados. La meta es proporcionar a los usuarios, independientemente de su experiencia previa en deporte, las herramientas necesarias para mejorar su rendimiento o iniciar un camino hacia el bienestar físico sin incurrir en una inversión temporal significativa en aprendizaje autodidacta o prácticas erróneas.

Como solución, he desarrollado FitFuelBalance, una plataforma integrada que ofrece servicios tanto en formato de aplicación web como móvil. Esta aplicación permite a los usuarios solicitar servicios personalizados de entrenadores y nutricionistas, quienes pueden utilizar un repositorio de ejercicios y dietas predefinidas para crear planes a medida. Los usuarios tienen la capacidad de revisar detalles, solicitar hacer ajustes en sus rutinas a los profesionales a través de la aplicación, garantizando así una experiencia personalizada y eficiente. Además, se incorpora el uso de inteligencia artificial para facilitar recomendaciones instantáneas y adaptaciones, reforzando la personalización de los planes ofrecidos.

El proyecto ha despertado un gran interés entre distintos perfiles de usuarios, desde expertos deportivos hasta individuos sin experiencia previa, como gente dedicada al entrenamiento o nutrición, demostrando el potencial de la tecnología para transformar la manera en que las personas acceden y gestionan su salud y bienestar.





\palabrasclave{Aplicación Web, Django, React, Nutrición, Deporte, Usuario, Ordenador, Móvil, Inteligencia Artificial}
