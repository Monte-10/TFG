La intersección entre el deporte y la nutrición es fundamental para mantener un estilo de vida saludable. Sin embargo, muchas personas enfrentan dificultades para integrar prácticas saludables en su rutina diaria debido a la falta de tiempo y conocimientos específicos. Este desafío a menudo conduce a la adopción de dietas genéricas o rutinas de ejercicio inadecuadas, lo que incrementa el riesgo de resultados negativos y lesiones. La principal barrera para adoptar un estilo de vida saludable radica en la falta de información relevante y accesible, así como en la limitación de tiempo para adquirir dichos conocimientos.

Reconociendo esta necesidad, mi Trabajo Final de Grado (TFG) se centra en simplificar el acceso a planes de nutrición y ejercicio personalizados. El objetivo es proporcionar a los usuarios, independientemente de su experiencia previa en deporte, las herramientas necesarias para mejorar su rendimiento o iniciar un camino hacia el bienestar físico, sin requerir una inversión significativa de tiempo en aprendizaje autodidacta o prácticas erróneas.

Como solución, he desarrollado FitFuelBalance, una plataforma integrada que ofrece servicios tanto en formato de aplicación web como móvil. Esta aplicación permite a los usuarios solicitar servicios personalizados de entrenadores y nutricionistas, quienes pueden utilizar un repositorio de ejercicios y dietas predefinidas para crear planes a medida. Los usuarios tienen la capacidad de revisar detalles y solicitar ajustes en sus rutinas a los profesionales a través de la aplicación, garantizando así una experiencia personalizada y eficiente. Además, se incorpora el uso de inteligencia artificial para facilitar recomendaciones instantáneas y adaptaciones, reforzando la personalización de los planes ofrecidos.

El proyecto ha despertado un gran interés entre distintos perfiles de usuarios, desde expertos deportivos hasta individuos sin experiencia previa, así como personas dedicadas al entrenamiento o la nutrición, demostrando el potencial de la tecnología para transformar la manera en que las personas acceden y gestionan su salud y bienestar.
Este documento detallará el análisis, diseño, codificación, pruebas e implementación de la plataforma FitFuelBalance, dirigida a empresas, instituciones y particulares que buscan mejorar su salud y rendimiento físico de manera eficiente y personalizada.


\palabrasclave{Aplicación Web, Django, React, Nutrición, Deporte, Usuario, Ordenador, Móvil, Inteligencia Artificial}
