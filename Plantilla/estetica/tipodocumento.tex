La intersección entre el deporte y la nutrición es esencial para mantener un estilo de vida saludable. No obstante, es difícil integrar prácticas saludables en una rutina diaria por falta de tiempo o conocimientos sobre el campo. Esto nos lleva a utilizar dietas o rutinas de ejercicio de internet o realizadas por gente no experta, incrementando el riesgo de resultados negativos.

La principal motivación para el desarrollo de FitFuelBalance radica en la necesidad de simplificar el acceso a planes personalizados de nutrición y ejercicio. La plataforma está diseñada para proporcionar a los usuarios las herramientas necesarias para mejorar su rendimiento físico o iniciar un camino hacia el bienestar, sin requerir una inversión significativa de tiempo en aprendizaje autodidacta o prácticas erróneas.

Estudios recientes han resaltado la importancia de una nutrición adecuada combinada con ejercicio regular para prevenir enfermedades crónicas y mejorar la calidad de vida. Según la Organización Mundial de la Salud (OMS), la inactividad física es uno de los principales factores de riesgo de mortalidad mundial, responsable de aproximadamente 3.2 millones de muertes cada año\cite{WHOPhysicalActivity}. También una dieta equilibrada puede prevenir hasta un 80\% de las enfermedades cardíacas prematuras y derrames cerebrales\cite{CDC}.

Además, un estudio publicado en el British Journal of Sports Medicine muestra que la combinación de actividad física y una dieta saludable reduce significativamente el riesgo de mortalidad por todas las causas. Según este estudio, las personas que realizan ejercicio regular y siguen una dieta equilibrada tienen un 29\% porciento menos de probabilidades de morir prematuramente en comparación con aquellas que no mantienen estos hábitos saludables\cite{BJSM}.

La plataforma FitFuelBalance no solo busca facilitar la adopción de hábitos saludables, sino también contribuir al bienestar general de la sociedad. Por ejemplo, un estudio realizado por el American College of Sports Medicine indica que las intervenciones tecnológicas en el ámbito de la salud pueden aumentar la adherencia a los programas de ejercicio y mejorar los resultados de salud\cite{ACSMRebrandX}.