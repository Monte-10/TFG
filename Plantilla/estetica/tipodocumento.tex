La idea del proyecto me fue propuesta por parte del Dr. Pablo Cerro, entonces se me explicó el gran alcance que podría tener esta aplicación y el avance que podría suponer en el mundo de la nutrición y el deporte. Al ser yo una persona que realiza mucho deporte y cuida su nutrición el proyecto llamó mi atención convirtiendose en mi trabajo de fin de grado.

La intersección entre el deporte y la nutrición es esencial para mantener un estilo de vida saludable. No obstante, es difícil integrar prácticas saludables en una rutina diaria por falta de tiempo o conocimientos sobre el campo. Esto nos lleva a utilizar dietas o rutinas de ejercicio de internet o realizadas por gente no experta, incrementando el riesgo de resultados negativos.

La principal motivación para el desarrollo de FitFuelBalance radica en la necesidad de simplificar el acceso a planes personalizados de nutrición y ejercicio. La plataforma está diseñada para proporcionar a los usuarios las herramientas necesarias para mejorar su rendimiento físico o iniciar un camino hacia el bienestar, sin requerir una inversión significativa de tiempo en aprendizaje autodidacta o prácticas erróneas.