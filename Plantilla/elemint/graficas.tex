El servidor se ha desarrollado utilizando el framework Django, que proporciona una estructura modular y reutilizable, facilitando el desarrollo y mantenimiento de la aplicación. A continuación, se describen los componentes principales de la estructura del servidor:

\begin{itemize}
    \item \textbf{Configuración del Proyecto:} El proyecto está configurado con un archivo \texttt{settings.py} que gestiona las configuraciones globales, incluyendo la base de datos, aplicaciones instaladas, middleware y otros ajustes importantes.
    
    \item \textbf{Aplicaciones Django:} Como se ha mencionado anteriormente, el servidor está dividido en varias aplicaciones Django, cada una responsable de una parte específica de la funcionalidad del sistema. Las aplicaciones principales incluyen:
    \begin{itemize}
        \item \textbf{Usuarios:} Maneja el registro, autenticación y gestión de perfiles de los usuarios.
        \item \textbf{Entrenamientos:} Gestiona la creación, actualización y seguimiento de los planes de entrenamiento.
        \item \textbf{Nutrición:} Administra los planes de nutrición y el seguimiento de la dieta de los usuarios.
    \end{itemize}
    
    \item \textbf{Modelos:} Los modelos de Django definen la estructura de la base de datos. Cada modelo se traduce en una tabla de la base de datos y define los campos y las relaciones entre ellos. Algunos de los modelos clave incluyen:
    \begin{itemize}
        \item \textbf{CustomUser:} Define la información básica del usuario. Este modelo deriva en \texttt{Trainer} y \texttt{RegularUser}, que representan a los entrenadores y nutricionistas, y usuarios normales respectivamente.
        \item \textbf{Diet:} Contiene todas las comidas, alimentos y recetas que se pueden utilizar en los planes de nutrición.
        \item \textbf{Option:} Contiene un plan de 3 opciones de comidas distintas para cada día durante una semana.
        \item \textbf{Training:} Se encarga de crear un entrenamiento con todos sus ejercicios correspondientes y sus series y repeticiones.
    \end{itemize}
    
    \item \textbf{Vistas:} Las vistas en Django son responsables de manejar las solicitudes HTTP y devolver las respuestas adecuadas. Utilizando Django REST Framework, las vistas se han estructurado como vistas basadas en clases (Class-Based Views), facilitando la reutilización y el manejo de lógica compleja. Las principales vistas incluyen:
    \begin{itemize}
        \item \textbf{Registro y Autenticación:} Maneja el registro de nuevos usuarios y la autenticación mediante JWT.
        \item \textbf{Gestión de Planes:} Permite a los entrenadores y nutricionistas crear, actualizar y eliminar planes de entrenamiento y nutrición.
        \item \textbf{Seguimiento de Progreso:} Controla la actualización de comidas y entrenamientos creados por los usuarios.
    \end{itemize}
    
    \item \textbf{Serializadores:} Los serializadores de Django REST Framework se utilizan para convertir instancias de modelos Django a formatos JSON y viceversa. Esto es esencial para la comunicación entre el servidor y los clientes (aplicaciones web y móvil). Los serializadores principales incluyen:
    \begin{itemize}
        \item \textbf{Serializadores de Usuario:} Convierten las instancias de los modelos de usuario.
        \item \textbf{Serializadores de Deporte:} Convierten las instancias de los modelos de la parte deportiva de la aplicación.
        \item \textbf{Serializadores de Nutrición:} Convierten las instancias de los modelos de la parte nutritiva de la aplicación.
    \end{itemize}
    
    \item \textbf{URLs:} Las rutas URL definen cómo se mapean las solicitudes HTTP a las vistas correspondientes. En el archivo \texttt{urls.py}, se han definido las rutas principales de la API, organizadas por aplicación.
    
    \item \textbf{Modelo de Datos:} La base de datos utilizada es PostgreSQL, elegida por su robustez y capacidad para manejar grandes volúmenes de datos de manera eficiente. Las migraciones de base de datos se gestionan mediante las herramientas integradas de Django.
    
    \item \textbf{API REST:} Se ha utilizado Django REST Framework para construir la API RESTful que permite la comunicación entre el servidor y los clientes (aplicaciones web y móvil). Esta API maneja las solicitudes HTTP y proporciona respuestas en formato JSON. Las llamadas a la API se autentican utilizando tokens JWT y la librería Axios en el frontend.
    
    \item \textbf{Autenticación y Autorización:} Se ha implementado un sistema de autenticación basado en tokens utilizando JWT (JSON Web Tokens) para asegurar las comunicaciones y gestionar las sesiones de usuario de manera segura.
    
    \item \textbf{Despliegue:} El servidor se ha desplegado en la plataforma Render, que permite gestionar de manera sencilla la infraestructura necesaria para la aplicación. Render proporciona un entorno de despliegue robusto y escalable, adecuado para las necesidades del proyecto.
\end{itemize}

Figura \ref{FIG:estructuraserver} ilustra la estructura general del servidor y la interacción entre sus componentes, así como la conexión con la API REST y los clientes (aplicaciones web y móvil):

\begin{figure}[Distribución Estructura Servidor]{FIG:estructuraserver}{Estructura del servidor de FitFuelBalance}
    \image{16cm}{}{estructuraserver}
\end{figure}