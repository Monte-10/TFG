El desarrollo de la API RESTful en FitFuelBalance se ha realizado utilizando Django REST Framework (DRF). Esta API permite la comunicación entre el servidor y los clientes (tanto la aplicación web como la móvil) a través de solicitudes HTTP, proporcionando respuestas en formato JSON. A continuación, se detalla la estructura y los componentes principales de la API.

\subsubsection{Endpoints y Rutas}
Las rutas de la API están definidas en el archivo \texttt{urls.py} del proyecto. Estas rutas mapean las solicitudes HTTP a las vistas correspondientes, las cuales manejan la lógica de negocio. Figura \ref{FIG:urls} muestra un ejemplo de rutas en Django.

\begin{figure}[Ejemplo Código Urls]{FIG:urls}{Ejemplo de rutas en Django}
\begin{verbatim}
from django.urls import path, include
from rest_framework.routers import DefaultRouter

router = DefaultRouter()
router.register(r'foods', views.FoodViewSet)
router.register(r'assigned_options', views.AssignedOptionViewSet)

urlpatterns = [
    path('', include(router.urls)),
    path('assign_option/', views.assignOption, name='assign_option'),
]
\end{verbatim}
\end{figure}

\subsubsection{Vistas y Controladores}
Las vistas en Django se encargan de manejar las solicitudes HTTP y devolver las respuestas correspondientes. En FitFuelBalance, las vistas se han implementado utilizando clases basadas en vistas (Class-Based Views) y ViewSets de DRF para facilitar la reutilización y el manejo de lógica compleja. Figura \ref{FIG:views} muestra un fragmento del código la vistas del proyecto en Django.

\begin{figure}[Ejemplo Código Vistas]{FIG:views}{Ejemplo de vistas en Django}

\begin{verbatim}
class FoodViewSet(viewsets.ModelViewSet):
    queryset = Food.objects.all()
    serializer_class = FoodSerializer
\end{verbatim}
\end{figure}

\subsubsection{Serializadores}
Los serializadores de DRF se utilizan para convertir instancias de modelos de Django a formatos JSON y viceversa. Esto es esencial para la comunicación entre el servidor y los clientes. Figura \ref{FIG:serializers} muestra un ejemplo de estos serializadores en Django.

\begin{figure}[Ejemplo Código Serializadores]{FIG:serializers}{Ejemplo de serializadores en Django}
\begin{verbatim}
class FoodSerializer(serializers.ModelSerializer):
    class Meta:
        model = Food
        fields = '__all__'
\end{verbatim}
\end{figure}

\newpage
\subsubsection{Métodos HTTP}
La API maneja los siguientes métodos HTTP:
\begin{itemize}
    \item \textbf{GET:} Solicita una representación de un recurso específico del servidor.
    \item \textbf{POST:} Envía un nuevo recurso al servidor.
    \item \textbf{PUT:} Envía un recurso existente al servidor con modificaciones.
    \item \textbf{DELETE:} Elimina un recurso del servidor.
\end{itemize}

\subsubsection{Ejemplo de Solicitud y Respuesta}
Figura \ref{FIG:getrequest} muestra un ejemplo de solicitud GET en Python utilizando la librería \texttt{requests}. En este caso, se solicita la lista de alimentos disponibles en la API de FitFuelBalance.
\begin{figure}[Ejemplo Solicitud GET]{FIG:getrequest}{Ejemplo de solicitud GET en Python}
\begin{verbatim}
response = requests.get(
    'https://fitfuelbalance.onrender.com/nutrition/foods/')
\end{verbatim}
\end{figure}

Figura \ref{FIG:getresponse} muestra un ejemplo de lo que podrías ser la respuesta a la solicitud GET anterior. En este caso, se devuelve una lista de alimentos en formato JSON con sus propiedades.

\begin{figure}[Ejemplo Respuesta GET]{FIG:getresponse}{Ejemplo de respuesta GET en JSON}
\begin{verbatim}
    [{
        "id": 1,
        "name": "Apple",
        "calories": 52,
        "protein": 0.3,
        "carbohydrates": 14,
        "sugar": 10,
        "fiber": 2.4,
        "fat": 0.2,
        "image": "url/media/foods/apple.jpg"
    },...]
\end{verbatim}
\end{figure}