El desarrollo de la API RESTful en FitFuelBalance se ha realizado utilizando Django REST Framework (DRF). Esta API permite la comunicación entre el servidor y los clientes (tanto la aplicación web como la móvil) a través de solicitudes HTTP, proporcionando respuestas en formato JSON. A continuación, se detalla la estructura y los componentes principales de la API.

\subsubsection{Endpoints y Rutas}
Las rutas de la API están definidas en el archivo \texttt{urls.py} del proyecto. Estas rutas mapean las solicitudes HTTP a las vistas correspondientes, las cuales manejan la lógica de negocio. Código \ref{COD:urls} muestra un ejemplo de rutas en Django.



\subsubsection{Vistas y Controladores}
Las vistas en Django se encargan de manejar las solicitudes HTTP y devolver las respuestas correspondientes. En FitFuelBalance, las vistas se han implementado utilizando clases basadas en vistas (Class-Based Views) y ViewSets de DRF para facilitar la reutilización y el manejo de lógica compleja. Código \ref{COD:views} muestra un fragmento del código la vistas del proyecto en Django.

\subsubsection{Serializadores}
Los serializadores de DRF se utilizan para convertir instancias de modelos de Django a formatos JSON y viceversa. Esto es esencial para la comunicación entre el servidor y los clientes. Código \ref{COD:serializers} muestra un ejemplo de estos serializadores en Django.

\subsubsection{Métodos HTTP}
La API maneja los siguientes métodos HTTP:
\begin{itemize}
    \item \textbf{GET:} Solicita una representación de un recurso específico del servidor.
    \item \textbf{POST:} Envía un nuevo recurso al servidor.
    \item \textbf{PUT:} Envía un recurso existente al servidor con modificaciones.
    \item \textbf{DELETE:} Elimina un recurso del servidor.
\end{itemize}

\subsubsection{Ejemplo de Solicitud y Respuesta}
Código \ref{COD:getrequest} muestra un ejemplo de solicitud GET en Python utilizando la librería \texttt{requests}. En este caso, se solicita la lista de alimentos disponibles en la API de FitFuelBalance.
\PythonCode[COD:getrequest]{Ejemplo Solicitud GET.}{Ejemplo de solicitud GET en Python.}{python/codigo.py}{18}{18}{1}

Código \ref{COD:getresponse} muestra un ejemplo de lo que podría ser la respuesta a la solicitud GET anterior. En este caso, se devuelve una lista de alimentos en formato JSON con sus propiedades.
\JavaCode[COD:getresponse]{Ejemplo de Respuesta GET.}{Ejemplo de respuesta GET en JSON.}{python/js.json}{1}{12}{1}