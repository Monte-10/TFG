El desarrollo de la API RESTful en FitFuelBalance se ha realizado utilizando Django REST Framework (DRF). Esta API permite la comunicación entre el servidor y los clientes (tanto la aplicación web como la móvil) a través de solicitudes HTTP, proporcionando respuestas en formato JSON. A continuación, se detalla la estructura y los componentes principales de la API.

\subsubsection{Endpoints y Rutas}
Las rutas de la API están definidas en el archivo \texttt{urls.py} del proyecto. Estas rutas mapean las solicitudes HTTP a las vistas correspondientes, las cuales manejan la lógica de negocio.

\begin{verbatim}
# urls.py

from django.urls import path, include
from rest_framework.routers import DefaultRouter
from . import views

router = DefaultRouter()
router.register(r'foods', views.FoodViewSet)
router.register(r'ingredients', views.IngredientViewSet)
router.register(r'dishes', views.DishViewSet)
router.register(r'meals', views.MealViewSet)
router.register(r'daily_diets', views.DailyDietViewSet)
router.register(r'diets', views.DietViewSet)
router.register(r'day_options', views.DayOptionViewSet)
router.register(r'week_options', views.WeekOptionViewSet)
router.register(r'options', views.OptionViewSet)
router.register(r'assigned_options', views.AssignedOptionViewSet)

urlpatterns = [
    path('', include(router.urls)),
    path('today_daily_diet/', views.TodayDailyDietView.as_view(), name='today_daily_diet'),
    path('daily_diet_by_date/<str:date>/', views.DailyDietByDateView.as_view(), name='daily_diet_by_date'),
    path('assign_option/', views.assignOption, name='assign_option'),
    path('adapt_diet_or_option/', views.adapt_diet_or_option, name='adapt_diet_or_option'),
]
\end{verbatim}

\subsubsection{Vistas y Controladores}
Las vistas en Django se encargan de manejar las solicitudes HTTP y devolver las respuestas correspondientes. En FitFuelBalance, las vistas se han implementado utilizando clases basadas en vistas (Class-Based Views) y ViewSets de DRF para facilitar la reutilización y el manejo de lógica compleja.

\begin{verbatim}
# views.py

from rest_framework import viewsets
from .models import *
from .serializers import *
from rest_framework.views import APIView
from rest_framework.response import Response
from rest_framework.decorators import api_view, permission_classes
from rest_framework.permissions import IsAuthenticated
from django.utils import timezone
import datetime

class FoodViewSet(viewsets.ModelViewSet):
    queryset = Food.objects.all()
    serializer_class = FoodSerializer

class IngredientViewSet(viewsets.ModelViewSet):
    queryset = Ingredient.objects.all()
    serializer_class = IngredientSerializer

class DishViewSet(viewsets.ModelViewSet):
    queryset = Dish.objects.all()
    serializer_class = DishSerializer

class MealViewSet(viewsets.ModelViewSet):
    queryset = Meal.objects.all()
    serializer_class = MealSerializer

class DailyDietViewSet(viewsets.ModelViewSet):
    queryset = DailyDiet.objects.all()
    serializer_class = DailyDietSerializer

class DietViewSet(viewsets.ModelViewSet):
    queryset = Diet.objects.all()
    serializer_class = DietSerializer

class TodayDailyDietView(APIView):
    permission_classes = [IsAuthenticated]
    
    def get(self, request, *args, **kwargs):
        today = timezone.now().date()
        user = request.user
        
        diets = Diet.objects.filter(user=user)
        today_diets = DailyDiet.objects.filter(diet__in=diets, date=today)
        
        serializer = DailyDietSerializer(today_diets, many=True)
        return Response(serializer.data)

@api_view(['POST'])
@permission_classes([IsAuthenticated])
def assignOption(request):
    user_id = request.data.get('userId')
    option_id = request.data.get('optionId')

    user = get_object_or_404(CustomUser, id=user_id)
    option = get_object_or_404(Option, id=option_id)

    assignment = AssignedOption.objects.create(
        user=user,
        option=option,
        start_date=timezone.now()
    )

    return Response({
        "message": "Option assigned successfully",
        "assignedOptionId": assignment.id,
        "optionId": option.id,
        "start_date": assignment.start_date.strftime("%Y-%m-%d")
    }, status=status.HTTP_201_CREATED)
\end{verbatim}

\subsubsection{Serializadores}
Los serializadores de DRF se utilizan para convertir instancias de modelos de Django a formatos JSON y viceversa. Esto es esencial para la comunicación entre el servidor y los clientes.

\begin{verbatim}
# serializers.py

from rest_framework import serializers
from .models import *

class FoodSerializer(serializers.ModelSerializer):
    class Meta:
        model = Food
        fields = '__all__'

class IngredientSerializer(serializers.ModelSerializer):
    food = serializers.PrimaryKeyRelatedField(queryset=Food.objects.all())
    
    class Meta:
        model = Ingredient
        fields = '__all__'

class DishSerializer(serializers.ModelSerializer):
    ingredients = IngredientSerializer(many=True)
    
    class Meta:
        model = Dish
        fields = '__all__'

class MealSerializer(serializers.ModelSerializer):
    dishes = DishSerializer(many=True)
    
    class Meta:
        model = Meal
        fields = '__all__'

class DailyDietSerializer(serializers.ModelSerializer):
    meals = MealSerializer(many=True)
    
    class Meta:
        model = DailyDiet
        fields = '__all__'

class DietSerializer(serializers.ModelSerializer):
    daily_diets = DailyDietSerializer(many=True)
    
    class Meta:
        model = Diet
        fields = '__all__'
\end{verbatim}

\subsubsection{Métodos HTTP}
La API maneja los siguientes métodos HTTP:
\begin{itemize}
    \item \textbf{GET:} Solicita una representación de un recurso específico del servidor.
    \item \textbf{POST:} Envía un nuevo recurso al servidor.
    \item \textbf{PUT:} Envía un recurso existente al servidor con modificaciones.
    \item \textbf{DELETE:} Elimina un recurso del servidor.
\end{itemize}

\subsubsection{Ejemplo de Solicitud y Respuesta}
Un ejemplo de solicitud GET para obtener todas las comidas:

\begin{verbatim}
import requests

response = requests.get('https://fitfuelbalance.onrender.com/nutrition/foods/')
print(response.json())
\end{verbatim}

Respuesta en formato JSON:

\begin{verbatim}
[
    {
        "id": 1,
        "name": "Apple",
        "calories": 52,
        "protein": 0.3,
        "carbohydrates": 14,
        "sugar": 10,
        "fiber": 2.4,
        "fat": 0.2,
        "image": "https://fitfuelbalance.onrender.com/media/foods/apple.jpg"
    },
    ...
]
\end{verbatim}