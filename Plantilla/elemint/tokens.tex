Para garantizar la seguridad y autenticidad de los usuarios en la plataforma FitFuelBalance, se ha implementado un sistema de autenticación basado en JSON Web Tokens (JWT). Este estándar de seguridad permite la comunicación segura entre dos partes mediante el uso de tokens firmados digitalmente. La implementación de JWT en este proyecto es crucial para asegurar que solo los usuarios registrados por un administrador puedan acceder a los recursos protegidos del sistema.

El proceso de autenticación con JWT en FitFuelBalance sigue los siguientes pasos: primero, el usuario ingresa sus credenciales en el formulario de inicio de sesión del front-end. Al enviar el formulario, se genera una solicitud HTTP POST al login. El controlador de inicio de sesión en el backend verifica las credenciales del usuario. Si son correctas, se genera un token firmado que contiene la identificación del usuario y otros datos relevantes. Este token se envía de vuelta al cliente y se almacena en el almacenamiento local del navegador.

Para cada solicitud posterior realizada por el cliente a recursos protegidos, el token se incluye en el encabezado de la solicitud HTTP. El middleware \texttt{TokenAuthentication} en el backend intercepta la solicitud y verifica la validez del token. Si el token es válido, se permite el acceso al recurso solicitado; de lo contrario, se devuelve una respuesta HTTP 401 Unauthorized.

A continuación, se muestra un diagrama del sistema de verificación de tokens:

DIAGRAMA TOKENS

\subsubsection{Middleware y Configuración}

El middleware \texttt{TokenAuthentication} es esencial para la autenticación de usuarios en cada solicitud. En el archivo \texttt{settings.py} se define la configuración necesaria para el uso de JWT en el proyecto Django:

\begin{verbatim}
REST_FRAMEWORK = {
    'DEFAULT_PERMISSION_CLASSES': [
        'rest_framework.permissions.IsAuthenticated',
    ],
    'DEFAULT_AUTHENTICATION_CLASSES': [
        'rest_framework.authentication.TokenAuthentication',
    ],
}
\end{verbatim}

En el archivo \texttt{views.py}, se define la vista para el manejo de la autenticación utilizando tokens JWT:

\begin{verbatim}
class LoginView(APIView):
    authentication_classes = []  # No authentication required
    permission_classes = []  # No permission required

    def post(self, request, *args, **kwargs):
        username = request.data.get('username')
        password = request.data.get('password')
        user = authenticate(username=username, password=password)
        if user is not None:
            token, created = Token.objects.get_or_create(user=user)
            return Response({'token': token.key, 'userId': user.id}, status=status.HTTP_200_OK)
        else:
            return Response({'detail': 'Invalid Credentials'}, status=status.HTTP_401_UNAUTHORIZED)
\end{verbatim}

\subsubsection{Uso de Tokens en el Front-End}

En el archivo \texttt{App.js} del Front-end, se maneja el almacenamiento del token y su uso en las solicitudes a la API:

\begin{verbatim}
useEffect(() => {
    if (authToken) {
        const fetchProfile = async () => {
            try {
                const response = await axios.get(`${process.env.REACT_APP_API_URL}/user/profile/`, {
                    headers: { 'Authorization': `Token ${localStorage.getItem('authToken')}` }
                });
                setProfile(response.data);
            } catch (error) {
                console.error("Error al cargar el perfil:", error);
            }
        };

        fetchProfile();
    }
}, [authToken]);

const handleLoginSuccess = (token, userId) => {
    localStorage.setItem('authToken', token);
    localStorage.setItem('userId', userId);
    setAuthToken(token);
};
\end{verbatim}

EJEMPLO TOKEN