La aplicación web de FitFuelBalance \circled{1} ha sido construida utilizando React.js, una biblioteca de JavaScript para construir interfaces de usuario. React permite crear aplicaciones web rápidas y eficientes, gracias a su enfoque en componentes reutilizables y su capacidad de manejar cambios de estado de manera eficiente. Los componentes principales definidos en la aplicación incluyen el componente de encabezado (Header), que incluye la navegación principal; el componente HomePage.js \circled{2}, que se encarga de la navegación a todos los componentes principales; y el componente App.js, que define las rutas y la estructura general de la aplicación. Para los estilos \circled{3}, se ha utilizado Bootstrap junto con CSS personalizado para asegurar una apariencia moderna y responsiva en toda la aplicación. Bootstrap proporciona una base sólida de componentes y utilidades que aceleran el desarrollo y aseguran la consistencia visual. La integración con el backend de Django se realiza a través de una serie de servicios definidos en la carpeta \texttt{components/} \circled{4}, los cuales utilizan \texttt{fetch} \circled{5} para realizar solicitudes HTTP y manejar las respuestas. El manejo del estado global de la aplicación se ha implementado utilizando \texttt{Context API} de React, lo que permite compartir datos y estados entre componentes sin necesidad de pasar props manualmente en múltiples niveles de la jerarquía de componentes. La autenticación de usuarios se maneja mediante JWT (JSON Web Tokens); al iniciar sesión, el backend proporciona un token que se almacena en el almacenamiento local del navegador y se adjunta a las cabeceras de las solicitudes HTTP subsecuentes para validar la identidad del usuario.