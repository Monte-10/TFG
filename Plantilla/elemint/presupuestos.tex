La aplicación web de FitFuelBalance ha sido construida utilizando React.js, una biblioteca de JavaScript para construir interfaces de usuario. React permite crear aplicaciones web rápidas y eficientes, gracias a su enfoque en componentes reutilizables y su capacidad de manejar cambios de estado de manera eficiente.

\begin{itemize}
    \item \textbf{Estructura del Proyecto:} La estructura del proyecto React.js se ha organizado de la siguiente manera, destacando los directorios y archivos más importantes:
    \begin{itemize}
        \item \texttt{src/} - Directorio principal del código fuente.
        \begin{itemize}
            \item \texttt{components/} - Contiene los componentes reutilizables de la UI.
            \begin{itemize}
              \item \texttt{nutrition/} - Componentes relacionados con la nutrición.
              \item \texttt{training/} - Componentes relacionados con el entrenamiento.
              \item \texttt{user/} - Componentes relacionados con el usuario.
              \item \texttt{HomePage.js} - Componente principal de la página de inicio.
            \end{itemize}
            \item \texttt{front/} - Contiene los componentes usados en la página principal.
            \item \texttt{css/} - Contiene los archivos CSS y de estilos.
            \item \texttt{App.js} - Archivo principal de la aplicación.
            \item \texttt{index.js} - Punto de entrada de la aplicación.
        \end{itemize}
    \end{itemize}

    \item \textbf{Componentes Principales:} Se han definido varios componentes clave que se utilizan en diferentes partes de la aplicación:
    \begin{itemize}
        \item \textbf{Header:} El componente de encabezado que incluye la navegación principal.
        \item \textbf{HomePage.js:} Se encarga de la navegación a todos los componentes principales.
        \item \textbf{App.js:} Componente principal de la aplicación que define las rutas y la estructura general.
    \end{itemize}

    \item \textbf{Estilos:} Se ha utilizado Bootstrap junto con CSS personalizado para asegurar una apariencia moderna y responsiva en toda la aplicación. Bootstrap proporciona una base sólida de componentes y utilidades que aceleran el desarrollo y aseguran la consistencia visual.

    \item \textbf{Integración con Backend:} La aplicación web se comunica con el backend de Django a través de una serie de servicios definidos en la carpeta \texttt{components/}. Estos servicios utilizan \texttt{fetch} para realizar solicitudes HTTP y manejar las respuestas. A continuación, se muestra un ejemplo de un servicio para obtener los planes de entrenamiento:
    \begin{verbatim}
        const apiUrl = process.env.REACT_APP_API_URL;

        export const getTrainingPlans = async () => {
            const response = await fetch(`${apiUrl}/sport/trainings/` {
                headers: {
                  'Authorization': `Token ${localStorage.getItem('token')}`
                }
            });
            const data = await response.json();
            return data;
        };
    \end{verbatim}

    \item \textbf{Manejo de Estado:} Para el manejo del estado global de la aplicación se ha utilizado \texttt{Context API} de React. Esto permite compartir datos y estados entre componentes sin necesidad de pasar props manualmente en múltiples niveles de la jerarquía de componentes.

    \item \textbf{Autenticación y Autorización:} La autenticación de usuarios se maneja mediante JWT (JSON Web Tokens). Al iniciar sesión, el backend proporciona un token que se almacena en el almacenamiento local del navegador. Este token se adjunta a las cabeceras de las solicitudes HTTP subsecuentes para validar la identidad del usuario.

\end{itemize}