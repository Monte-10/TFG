El desarrollo del Back-End se ha llevado a cabo utilizando Django, un framework de desarrollo web de alto nivel en Python. Django permite la creación de aplicaciones web seguras y mantenibles de manera rápida y eficiente. A continuación se detalla la estructura y los componentes principales del servidor:
\begin{enumerate}
    \item \textbf{Configuración del Proyecto:} Gestión de configuraciones globales mediante el archivo \texttt{settings.py}.
    \item \textbf{Aplicaciones Django:} División del servidor en varias aplicaciones responsables de funcionalidades específicas.
    \item \textbf{Modelos:} Definición de la estructura de la base de datos con modelos de Django.
    \item \textbf{Vistas:} Manejo de solicitudes HTTP y respuestas con Django REST Framework.
    \item \textbf{Serializadores:} Conversión de instancias de modelos Django a formatos JSON y viceversa.
    \item \textbf{URLs:} Definición de rutas URL para mapear solicitudes HTTP a las vistas correspondientes.
    \item \textbf{API REST:} Construcción de una API RESTful para la comunicación entre el servidor y los clientes.
    \item \textbf{Autenticación y Autorización:} Implementación de un sistema de autenticación basado en tokens JWT.
    \item \textbf{Despliegue:} Despliegue del servidor en la plataforma Render.
\end{enumerate}

En el Apéndice \ref{CAP:FUNCENT}, Figura \ref{FIG:backend}, se muestra un diagrama de componentes del Back-End.