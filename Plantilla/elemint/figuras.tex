Esta sección detalla el desarrollo del proyecto, explicando en detalle los procedimientos seguidos y las tecnologías utilizadas. El sistema se ha dividido en dos partes principales: Back-End y Front-End. Se proporcionará una descripción detallada de cada una de estas partes junto con las herramientas y tecnologías empleadas. 

\subsection{Back-End}

El desarrollo del Back-End se ha llevado a cabo utilizando Django, un framework de desarrollo web de alto nivel en Python. Django permite la creación de aplicaciones web seguras y mantenibles de manera rápida y eficiente. A continuación se detalla la estructura y los componentes principales del servidor:

\subsubsection{Estructura del Servidor}

El servidor ha sido diseñado para gestionar las solicitudes de los usuarios, manejar la lógica de negocio y comunicarse con la base de datos PostgreSQL. Los componentes clave del servidor incluyen:

\begin{itemize}
    \item \textbf{Modelo de Datos:} La base de datos utilizada es PostgreSQL, elegida por su robustez y capacidad para manejar grandes volúmenes de datos de manera eficiente. Las migraciones de base de datos se gestionan mediante las herramientas integradas de Django.
    
    \item \textbf{Aplicaciones:} Tres aplicaciones principales han sido desarrolladas para dividir la lógica, estas son: usuarios, alimentación y deporte.
    
    \item \textbf{API REST:} Se ha utilizado Django REST Framework para construir la API RESTful que permite la comunicación entre el servidor y los clientes (aplicaciones web y móvil). Esta API maneja las solicitudes HTTP y proporciona respuestas en formato JSON.
    
    \item \textbf{Autenticación y Autorización:} Se ha implementado un sistema de autenticación basado en tokens utilizando JWT (JSON Web Tokens) para asegurar las comunicaciones y gestionar las sesiones de usuario de manera segura.
    
    \item \textbf{Despliegue:} El servidor se ha desplegado en la plataforma Render, que permite gestionar de manera sencilla la infraestructura necesaria para la aplicación. Render proporciona un entorno de despliegue robusto y escalable, adecuado para las necesidades del proyecto.
    
\end{itemize}

\subsubsection{Base de Datos}

La base de datos PostgreSQL ha sido gestionada utilizando Neon, una herramienta que facilita la configuración y el manejo de bases de datos PostgreSQL en la nube. Las tablas principales y sus relaciones son las siguientes:

FIGURA DE LA BASE DE DATOS.

La estructura de la base de datos ha sido diseñada para garantizar la integridad de los datos y facilitar el acceso eficiente a la información necesaria para las funcionalidades de la aplicación.

\subsubsection{Despliegue y Gestión de la Aplicación}

El proceso de despliegue del servidor ha sido gestionado mediante Git y GitHub, utilizando un flujo de trabajo basado en ramas para controlar las versiones del código y facilitar la colaboración. El despliegue en Render ha sido probado siempre, asegurando que cada cambio en el código sea probado y desplegado de manera segura y eficiente.

El siguiente diagrama muestra la arquitectura general del Back-End y su interacción con la base de datos y los clientes.

FIGURA DE LA ARQUITECTURA DEL BACK-END.

En las siguientes secciones se detallará el desarrollo específico de cada componente del Back-End, proporcionando ejemplos de código y explicaciones de las decisiones tomadas durante el proceso de desarrollo.