El desarrollo del Back-End se ha llevado a cabo utilizando Django, un framework de desarrollo web de alto nivel en Python. Django permite la creación de aplicaciones web seguras y mantenibles de manera rápida y eficiente. A continuación se detalla la estructura y los componentes principales del servidor: 

La configuración del proyecto se realiza mediante la gestión de configuraciones globales con el archivo \texttt{settings.py}. El servidor está dividido en varias aplicaciones Django, cada una responsable de funcionalidades específicas. Los modelos, que definen la estructura de la base de datos utilizando modelos de Django. Las vistas, que manejan las solicitudes HTTP y las respuestas con Django REST Framework. Los serializadores, los cuales convierten instancias de modelos Django a formatos JSON y viceversa. Las URLs, que se definen para mapear solicitudes HTTP a las vistas correspondientes. La API REST, que se construye para permitir la comunicación entre el servidor y los clientes. La autenticación y autorización, la cual se implementan mediante un sistema basado en tokens JWT y finalmente, el despliegue del servidor que se realiza en la plataforma Render.

En el Apéndice \ref{CAP:FUNCENT}, Figura \ref{FIG:backend}, se muestra un diagrama de componentes del Back-End.
