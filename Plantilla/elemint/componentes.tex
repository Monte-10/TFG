En esta sección se detallarán los principales componentes, estos son los elementos que componen la interfaz de usuario de la aplicación web de FitFuelBalance. Cada componente tiene una función específica y contribuye a la funcionalidad general de la aplicación. A continuación se describen los componentes más importantes y su función en la aplicación.
\subsubsection{App.js}
El componente principal \texttt{App.js} \circled{6} es el núcleo de la aplicación web. En este archivo se define la estructura general de la aplicación, incluyendo la barra de navegación, las rutas y la gestión del estado de autenticación del usuario. Utiliza \texttt{useState} para gestionar el estado de autenticación del usuario, almacenando el token de autenticación en el \texttt{localStorage}. Utiliza \texttt{useEffect} para obtener el perfil del usuario desde el backend al cargar el componente si existe un token de autenticación válido. Maneja el inicio y cierre de sesión, almacenando y eliminando el token de autenticación en el \texttt{localStorage}. Además, permite buscar entrenadores basados en especialidades y tipo de entrenador, enviando una solicitud HTTP al backend y actualizando el estado con los resultados. También envía solicitudes a entrenadores específicos mediante una solicitud HTTP POST al backend.

El componente \texttt{App} se organiza en varias secciones clave: la barra de navegación, implementada utilizando componentes de \texttt{react-bootstrap}, permite al usuario acceder rápidamente a las diferentes secciones de la aplicación, como Nutrición y Deporte. Utiliza \texttt{react-router-dom} para definir las rutas de la aplicación. Cada ruta está asociada a un componente específico que se renderiza cuando el usuario navega a esa ruta. Las principales vistas y componentes incluyen \texttt{HomePage}, \texttt{Login}, \texttt{Profile}, entre otros.

\subsubsection{Componentes de Nutrición}
Los componentes de la sección de nutrición \circled{7} en la aplicación web de FitFuelBalance son fundamentales para gestionar y administrar los planes de nutrición. Los componentes clave incluyen \texttt{CreateDiet.js}, que permite a los nutricionistas crear planes de dieta completos; \texttt{CreateDish.js}, utilizado para crear platos individuales que se pueden incluir en los planes de dieta; \texttt{EditDiet.js}, que facilita la edición de planes de dieta existentes; y \texttt{ListFood.js}, que presenta una lista completa de todos los alimentos en la base de datos, con funcionalidades de búsqueda y filtrado. Estos componentes permiten a los nutricionistas definir comidas específicas, ajustar porciones y modificar ingredientes según las necesidades del usuario.

\subsubsection{Componentes de Deporte}
Los componentes de deporte \circled{8} en la aplicación web están diseñados para gestionar y visualizar los diferentes aspectos relacionados con los ejercicios y entrenamientos. Los componentes clave incluyen \texttt{CreateExercise.js}, que permite a los entrenadores crear nuevos ejercicios; \texttt{CreateTraining.js}, utilizado para crear planes de entrenamiento completos; \texttt{EditExercise.js}, que permite a los entrenadores editar los detalles de un ejercicio existente; y \texttt{ListExercise.js}, que proporciona una lista de todos los ejercicios disponibles en la base de datos. Estos componentes permiten a los entrenadores definir, ajustar y visualizar ejercicios y planes de entrenamiento.

\subsubsection{Componentes de Usuario}
Los componentes relacionados con la gestión de usuarios \circled{9} en FitFuelBalance son esenciales para el manejo de la autenticación y los perfiles de usuario. El componente \texttt{Login.js} maneja el inicio de sesión de los usuarios, verificando las credenciales y gestionando los tokens JWT. \texttt{ManageClients.js} proporciona una interfaz para que los entrenadores gestionen la lista de sus clientes, actualizando información y asignando planes de entrenamiento o dietas. \texttt{Profile.js} permite a los usuarios ver y actualizar su perfil personal. Otros componentes importantes incluyen \texttt{RegularSignUp.js}, para el registro de nuevos usuarios normales, y \texttt{SearchTrainer.js}, que permite a los usuarios buscar entrenadores y nutricionistas en la plataforma.
