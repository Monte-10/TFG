En esta sección se detallarán los principales componentes, estos son los elementos que componen la interfaz de usuario de la aplicación web de FitFuelBalance. Cada componente tiene una función específica y contribuye a la funcionalidad general de la aplicación. A continuación se describen los componentes más importantes y su función en la aplicación.
\subsubsection{App.js}
El componente principal \texttt{App.js} \circled{6} es el núcleo de la aplicación web. En este archivo se define la estructura general de la aplicación, incluyendo la barra de navegación, las rutas y la gestión del estado de autenticación del usuario. Utiliza \texttt{useState} para gestionar el estado de autenticación del usuario, almacenando el token de autenticación en el \texttt{localStorage}. Utiliza \texttt{useEffect} para obtener el perfil del usuario desde el backend al cargar el componente si existe un token de autenticación válido. Maneja el inicio y cierre de sesión, almacenando y eliminando el token de autenticación en el \texttt{localStorage}. Además, permite buscar entrenadores basados en especialidades y tipo de entrenador, enviando una solicitud HTTP al backend y actualizando el estado con los resultados. También envía solicitudes a entrenadores específicos mediante una solicitud HTTP POST al backend.

El componente \texttt{App} se organiza en varias secciones clave: la barra de navegación, implementada utilizando componentes de \texttt{react-bootstrap}, permite al usuario acceder rápidamente a las diferentes secciones de la aplicación, como Nutrición y Deporte. Utiliza \texttt{react-router-dom} para definir las rutas de la aplicación. Cada ruta está asociada a un componente específico que se renderiza cuando el usuario navega a esa ruta. Las vistas y componentes incluyen \texttt{HomePage}, \texttt{Login}, \texttt{Profile}.

\subsubsection{Componentes de Nutrición}
Los componentes de la sección de nutrición \circled{7} en la aplicación web de FitFuelBalance son fundamentales para gestionar y administrar los planes de nutrición. La aplicación permite la administración de estos planes gracias a la capacidad de crear dietas para usuarios entre fechas concretas. Sin embargo, su función principal para administrar estos planes se encuentra en las \texttt{Opciones}, que permiten la realización de un plan semanal con tres opciones a elegir para que el usuario tenga más libertad de elección y no se sienta limitado.

Los componentes clave incluyen \texttt{CreateDiet.js} para crear planes de dieta completos, \\\texttt{CreateDish.js} para crear platos individuales, \texttt{EditDiet.js} para editar planes existentes y \\\texttt{ListFood.js} para listar alimentos con funcionalidades de búsqueda y filtrado. Estos componentes permiten definir comidas específicas, ajustar porciones y modificar ingredientes según las necesidades del usuario. Además, tras la creación de un plan con opciones, se genera automáticamente un PDF para facilitar el seguimiento del plan para el usuario.

Para simplificar aún más el trabajo de los nutricionistas, se ha implementado una función llamada \texttt{AdaptOption}. Esta función permite introducir un usuario y una opción; el componente recoge los datos del usuario y, mediante un cálculo basado en la fórmula de Harris-Benedict, adapta el plan de nutrición a las necesidades específicas del usuario. La fórmula de Harris-Benedict estima la Tasa Metabólica Basal (BMR) o el gasto energético en reposo en función del peso, la altura, la edad y el género del individuo. Para hombres: \(\text{BMR} = 88.362 + (13.397 \times \text{peso en kg}) + (4.799 \times \text{altura en cm}) - (5.677 \times \text{edad en años})\) y para mujeres: \(\text{BMR} = 447.593 + (9.247 \times \text{peso en kg}) + (3.098 \times \text{altura en cm}) - (4.330 \times \text{edad en años})\). Este valor se multiplica por un factor de actividad para obtener el requerimiento energético diario. Por ejemplo, una mujer de 30 años, 70 kg y 165 cm, con un nivel de actividad moderado, tendría un BMR de 1472.84 kcal/día y un requerimiento energético diario de 2283.90 kcal/día. El fragmento más relevante del funcionamiento de esta función se muestra en el Código \ref{COD:adaptoption}.

\subsubsection{Componentes de Deporte}
Los componentes de deporte \circled{8} en la aplicación web están diseñados para gestionar y visualizar los diferentes aspectos relacionados con los ejercicios y entrenamientos. Los componentes clave incluyen \texttt{CreateExercise.js}, que permite a los entrenadores crear nuevos ejercicios; \\\texttt{CreateTraining.js}, utilizado para crear planes de entrenamiento completos; \texttt{EditExercise.js}, que permite a los entrenadores editar los detalles de un ejercicio existente; y \texttt{ListExercise.js}, que proporciona una lista de todos los ejercicios disponibles en la base de datos. Gracias a estos componentes los entrenadores pueden crear planes de entrenamiento personalizados para sus clientes, incluyendo ejercicios específicos, series y repeticiones, modificando la intensidad y la duración según las necesidades del usuario. Además todos estos entrenamiento pueden seguirse mientras se realizan en la aplicación móvil. Como anteriormente, un pdf se genera para facilitar el seguimiento del plan.

\subsubsection{Componentes de Usuario}
Los componentes relacionados con la gestión de usuarios \circled{9} en FitFuelBalance son esenciales para el manejo de la autenticación y los perfiles de usuario. El componente \texttt{Login.js} maneja el inicio de sesión de los usuarios, verificando las credenciales y gestionando los tokens JWT. \texttt{ManageClients.js} proporciona una interfaz para que los entrenadores gestionen la lista de sus clientes, actualizando información y asignando planes de entrenamiento o dietas. \texttt{Profile.js} permite a los usuarios ver y actualizar su perfil personal. También destacan otros componentes para el registro de usuarios o la búsqueda de entrenadores.

Ahora, el componente que más cabe destacar es \texttt{Profile.js}, el cuál contiene toda la información del usuario y sus medidas, las cuales se pueden actualizar en cualquier momento. Además, se puede ver el progreso del usuario en la aplicación, ya que se muestra un gráfico con la evolución de las medidas del usuario a lo largo del tiempo. Los entrenadores también pueden ver el progreso de sus clientes, lo que les permite ajustar los planes de entrenamiento o dietas según sea necesario. Figura \ref{FIG:progress} muestra un ejemplo de este gráfico de progreso.

\begin{figure}[Gráfico de Progreso]{FIG:progress}{Gráfico de Progreso en FitFuelBalance}
  \image{7cm}{}{grafico1}
  \image{7cm}{}{grafico2}
\end{figure}