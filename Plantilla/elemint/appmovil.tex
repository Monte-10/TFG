La aplicación móvil \circled{10} de FitFuelBalance ha sido desarrollada utilizando React Native, un framework que permite crear aplicaciones móviles nativas para iOS y Android utilizando JavaScript y React. Esta tecnología ha sido seleccionada por su capacidad de compartir código entre plataformas y su excelente rendimiento en dispositivos móviles. La Figura \ref{FIG:estrucmovil} del Apéndice \ref{CAP:FUNCENT} muestra la estructura general de la aplicación móvil.

La navegación \circled{11} en la aplicación móvil se ha implementado utilizando React Navigation. Esta biblioteca facilita la gestión de la navegación entre pantallas y la organización de la estructura de navegación de la aplicación. Código \ref{COD:navconfig} muestra un ejemplo de configuración de navegación.

Las pantallas de la aplicación móvil se encuentran en la carpeta \texttt{screens/} \circled{12} y contienen los componentes principales de la aplicación. Algunas de las pantallas más importantes son: \\\texttt{HomeScreen.js}, \texttt{CalendarScreen.js} y \texttt{LoginScreen.js}, Figura \ref{FIG:calendarscreen} muestra estas pantallas además de \texttt{TodayScreen.js}. Estas pantallas se encargan de mostrar la información y permitir la interacción del usuario con la aplicación.

Para los estilos \circled{13} de la aplicación móvil se ha utilizado una combinación de hojas de estilo en JavaScript y StyleSheet de React Native. Esto permite definir estilos que se aplican de manera consistente en toda la aplicación y aprovechar características específicas de la plataforma.

Similar a la aplicación web, la aplicación móvil se comunica con el backend de Django a través de servicios definidos en la carpeta \texttt{api/}. Estos servicios utilizan \texttt{fetch} \circled{14} para realizar solicitudes HTTP y manejar las respuestas. Código \ref{COD:reactnativeservice} muestra un ejemplo de un servicio para obtener planes de entrenamiento.

\begin{figure}[CalendarScreen, TodayScreen, LoginScreen y HomeScreen]{FIG:calendarscreen}{Pantallas CalendarScreen, TodayScreen, LoginScreen y HomeScreen de la aplicación móvil.}
    \image{5cm}{}{CalendarHomeScreens}
    \image{5cm}{}{TodayScreen}
    \image{5cm}{}{LoginScreen}
\end{figure}

\newpage