La aplicación móvil de FitFuelBalance ha sido desarrollada utilizando React Native, un framework que permite crear aplicaciones móviles nativas para iOS y Android utilizando JavaScript y React. Esta tecnología ha sido seleccionada por su capacidad de compartir código entre plataformas y su excelente rendimiento en dispositivos móviles.

\begin{itemize}
    \item \textbf{Estructura del Proyecto:} La estructura del proyecto React Native se ha organizado de la siguiente manera:
    \begin{itemize}
        \item \texttt{src/} - Directorio principal del código fuente.
        \begin{itemize}
            \item \texttt{api/} - Contiene los archivos relacionados con la API.
            \begin{itemize}
                \item \texttt{trainingApi.js} - Archivo para las llamadas a la API relacionadas con los entrenamientos.
            \end{itemize}
            \item \texttt{AuthContext.js} - Contexto de autenticación.
            \item \texttt{components/} - Contiene los componentes reutilizables de la UI.
            \begin{itemize}
                \item \texttt{CountdownTimer.js} - Componente para un temporizador de cuenta regresiva.
            \end{itemize}
            \item \texttt{navigation/} - Configuración de la navegación de la aplicación.
            \begin{itemize}
                \item \texttt{AppNavigator.js} - Archivo de configuración del navegador de la aplicación.
            \end{itemize}
            \item \texttt{screens/} - Contiene las diferentes pantallas de la aplicación, la más importantes son:
            \begin{itemize}
                \item \texttt{ActiveTrainingScreen.js} - Pantalla de entrenamiento activo.
                \item \texttt{CalendarScreen.js} - Pantalla del calendario.
                \item \texttt{FoodDetailsScreen.js} - Pantalla de detalles de alimentos.
                \item \texttt{HomeScreen.js} - Pantalla de inicio.
                \item \texttt{LoginScreen.js} - Pantalla de inicio de sesión.
                \item \texttt{TodayScreen.js} - Muestra los entrenamientos y comidas del día.
                \item \texttt{TrainingDetailsScreen.js} - Pantalla de detalles del entrenamiento.
            \end{itemize}
        \end{itemize}
    \end{itemize}

    \item \textbf{Navegación:} La navegación en la aplicación móvil se ha implementado utilizando React Navigation. Esta biblioteca facilita la gestión de la navegación entre pantallas y la organización de la estructura de navegación de la aplicación. Figura \ref{FIG:navconfig} muestra un ejemplo de configuración de navegación.
    \begin{figure}[Configuración de Navegación]{FIG:navconfig}{Configuración de navegación en React Navigation}
    \begin{verbatim}
        const AppNavigator = () => {
            return (
                <Stack.Navigator initialRouteName="LoginScreen">
                    <Stack.Screen name="HomeScreen" component={HomeScreen} />
                    <Stack.Screen name="LoginScreen" component={LoginScreen} />
                    <Stack.Screen name="TrainingDetailsScreen" component={TrainingDetailsScreen} />
                </Stack.Navigator>
            );
        }
        export default AppNavigator;
    \end{verbatim}
    \end{figure}

    \item \textbf{Estilos:} Para los estilos de la aplicación móvil se ha utilizado una combinación de hojas de estilo en JavaScript y StyleSheet de React Native. Esto permite definir estilos que se aplican de manera consistente en toda la aplicación y aprovechar características específicas de la plataforma.

    \item \textbf{Integración con Backend:} Similar a la aplicación web, la aplicación móvil se comunica con el backend de Django a través de servicios definidos en la carpeta \texttt{api/}. Estos servicios utilizan \texttt{fetch} para realizar solicitudes HTTP y manejar las respuestas. Figura \ref{FIG:reactnativeservice} muestra un ejemplo de un servicio para obtener planes de entrenamiento.
    \begin{figure}[Ejemplo Servicio React Native]{FIG:reactnativeservice}{Ejemplo de servicio en React Native para obtener planes de entrenamiento}
    \begin{verbatim}
        import { API_URL } from '../config';
        export const fetchTrainingPlans = async () => {
            try {
                const response = await fetch(`${API_URL}/training-plans/`);
                if (!response.ok) throw new Error('Network response was not ok');
                const data = await response.json();
                return data;
            } catch (error) {
                console.error('Error fetching training plans:', error);
                return [];
            }
        };
    \end{verbatim}
    \end{figure}

\end{itemize}
