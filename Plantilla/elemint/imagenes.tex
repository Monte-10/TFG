Para la persistencia de datos en el proyecto FitFuelBalance, se ha utilizado PostgreSQL, un sistema de gestión de bases de datos relacional conocido por su robustez, flexibilidad y soporte para tipos de datos avanzados. A continuación, se describen los aspectos clave de la estructura de la base de datos y su integración con Django.

\begin{itemize}
    \item \textbf{Diseño de la Base de Datos:} El diseño de la base de datos ha sido cuidadosamente planificado para asegurar la eficiencia y escalabilidad. Los principales esquemas de la base de datos incluyen las siguientes tablas:
    \begin{itemize}
        \item \textbf{CustomUser:} Almacena información de los usuarios, incluyendo su nombre, correo electrónico, contraseña (hasheada), rol (usuario normal, entrenador/nutricionista, administrador) y otros detalles relevantes.
        \item \textbf{Training:} Registra los planes de entrenamiento creados por los entrenadores, incluyendo el nombre del plan, la descripción y los ejercicios asociados.
        \item \textbf{Diet:} Almacena los planes de dietas desarrollados por los nutricionistas, con detalles sobre las comidas, porciones, asi como su duración.
        \item \textbf{Option:} Registra las opciones de comidas para cada día de la semana, permitiendo a los usuarios seleccionar entre diferentes alternativas.
    \end{itemize}

    \item \textbf{Migraciones:} Django facilita la gestión de cambios en la estructura de la base de datos a través del sistema de migraciones. Cada vez que se realiza un cambio en los modelos, se crea una migración que, al aplicarse, sincroniza la base de datos con los nuevos cambios. Esto asegura que la base de datos esté siempre en línea con el código del proyecto.

    \item \textbf{ORM de Django:} Para interactuar con la base de datos, se utiliza el Object-Relational Mapping (ORM) de Django, que permite trabajar con la base de datos utilizando objetos Python en lugar de escribir consultas SQL directamente. Esto no solo simplifica el código sino que también mejora la portabilidad y mantenimiento del mismo. Por ejemplo, una consulta para obtener todos los planes de entrenamiento podría verse así:
    \begin{verbatim}
        trainings = Training.objects.all()
    \end{verbatim}

    \item \textbf{Conexión a la Base de Datos:} La configuración de la conexión a la base de datos se encuentra en el archivo \texttt{settings.py} del proyecto Django, donde se especifican los detalles necesarios para conectarse a PostgreSQL:
    \begin{verbatim}
        DATABASES = {
            'default': {
                'ENGINE': 'django.db.backends.postgresql',
                'NAME': 'fitfuelbalance_db',
                'USER': 'dbuser',
                'PASSWORD': 'password',
                'HOST': 'localhost',
                'PORT': '5432',
            }
        }
    \end{verbatim}

    \item \textbf{Gestión de Datos:} Para la administración y gestión de la base de datos, se ha utilizado Neon, una herramienta que facilita la visualización y manipulación de datos de PostgreSQL. Neon ofrece una interfaz gráfica intuitiva que permite realizar consultas, revisar tablas y gestionar esquemas de manera eficiente.

    \item \textbf{Seguridad y Copias de Seguridad:} Se han implementado medidas de seguridad para proteger los datos almacenados. Esto incluye la encriptación de contraseñas de usuarios utilizando el sistema de hash de Django y la realización periódica de copias de seguridad de la base de datos para prevenir pérdidas de datos.

\end{itemize}

La elección de PostgreSQL y la integración con Django ha permitido una gestión de datos eficiente y segura, asegurando que la aplicación pueda escalar y manejar grandes volúmenes de datos de manera efectiva.