El desarrollo de FitFuelBalance ha requerido el uso de diversas herramientas y entornos para facilitar el proceso de programación y despliegue. A continuación, se describen las principales herramientas y entornos utilizados:

\subsubsection{Control de Versiones}

Para el control de versiones se ha utilizado Git junto con la plataforma GitHub. Git es un sistema de control de versiones distribuido que permite a los desarrolladores realizar un seguimiento de los cambios en el código fuente y colaborar de manera eficiente. GitHub, por su parte, proporciona un entorno de alojamiento de repositorios Git con características adicionales como la gestión de issues, pull requests y la integración continua. El código fuente del proyecto se encuentra alojado en un repositorio privado en GitHub, lo que permite realizar un seguimiento detallado de los cambios, gestionar versiones y colaborar con otros desarrolladores de manera eficiente. Se han utilizado ramas (branches) para organizar el desarrollo de nuevas funcionalidades y la corrección de errores. La rama \texttt{main} contiene la versión estable del proyecto, mientras que las ramas que contienen la aplicación en producción (\texttt{primer-deploy/}), versiones anteriores estables (\texttt{beforeRemake/}) y versiones para probar funcionalidades nuevas (\texttt{modelos-unicos/}) se utilizan para trabajar en paralelo sin afectar la versión principal.

\subsubsection{Entorno de Desarrollo}

Para el desarrollo del proyecto se ha utilizado el editor de código Visual Studio Code, una herramienta potente y versátil que ofrece una amplia gama de extensiones y características para facilitar la programación.

\subsubsection{Diseño de la Interfaz de Usuario}

Para el diseño de la interfaz de usuario se ha utilizado Figma, una herramienta de diseño colaborativa que permite crear prototipos y diseños de alta fidelidad.

\subsubsection{Base de Datos}

Para la gestión de la base de datos se ha utilizado Neon, una plataforma basada en PostgreSQL que facilita la creación y administración de bases de datos en la nube.

\subsubsection{Despliegue}

El despliegue de las aplicaciones web y backend se ha realizado utilizando Render, una plataforma de despliegue en la nube que facilita el proceso de publicación de aplicaciones. Esta plataforma ha sido seleccionada por su simplicidad y eficiencia en el despliegue de aplicaciones. Render permite automatizar el despliegue a partir del código fuente alojado en GitHub, asegurando que las últimas versiones del proyecto estén siempre disponibles en producción.

\subsubsection{Testing y Depuración}

Durante el desarrollo del proyecto, se han utilizado diversas herramientas y técnicas para realizar pruebas y depuración de código. React Native Debugger \cite{ReactNativeDebugger} se ha utilizado para depurar la aplicación móvil desarrollada en React Native. Postman \cite{Postman} ha sido la herramienta utilizada para probar las API desarrolladas en Django y asegurarse de que respondan correctamente a las solicitudes. Pytest \cite{Pytest} es el framework de pruebas para Python, utilizado para crear y ejecutar pruebas automatizadas del backend. Google Chrome DevTools \cite{GoogleChromeDevTools} ha sido la herramienta de desarrollo integrada en el navegador Chrome.

\subsubsection{Gestión de Dependencias}

Para gestionar las dependencias del proyecto se han utilizado herramientas específicas para cada entorno de desarrollo. \texttt{pip} \cite{pip} se ha utilizado para gestionar las dependencias del backend desarrollado en Python y Django. \texttt{npm} \cite{npm} se ha utilizado para gestionar las dependencias del frontend desarrollado en React y React Native.