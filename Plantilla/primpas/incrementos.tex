Los incrementos se dividen en Back-end, Front-end y Aplicación Móvil. Generalmente, un incremento dentro de una sección no comienza hasta que el incremento anterior de la misma sección se ha completado. Sin embargo, en algunos casos, debido a problemas de dependencia lógica o por razones de prueba, puede haber dos o más incrementos en desarrollo simultáneamente. Por la misma razón, a veces se han desarrollado incrementos de más de una sección al mismo tiempo.

Cada incremento tiene un pequeño ciclo de vida que consiste en un análisis, un diseño, desarrollo y finalmente pruebas. En esta sección, solo veremos la definición de cada incremento sin entrar en detalles sobre su ciclo de vida.

\subsubsection{Back-end}
\begin{itemize}
    \item \textbf{IN1.1} Crear una base de datos PostgreSQL y configurar el servidor HTTP local con Django.
    \item \textbf{IN1.2} Definir modelos en Django para gestionar la parte de deporte.
    \item \textbf{IN1.3} Definir modelos en Django para gestionar la parte de nutrición.
    \item \textbf{IN1.4} Definir modelos en Django para gestionar los usuarios.
    \item \textbf{IN1.5} Definir vistas para controlar las operaciones de los modelos.
    \item \textbf{IN1.6} Crear controladores para manejar los inicios de sesión y registros de usuarios.
    \item \textbf{IN1.7} Crear APIs RESTful con Django REST framework para manejar las llamadas de los clientes.
    \item \textbf{IN1.8} Implementar la lógica de negocio y middleware necesarios para las operaciones del servidor.
    \item \textbf{IN1.9} Desplegar todo en Render.
\end{itemize}

\subsubsection{Front-end}
\begin{itemize}
    \item \textbf{IN2.1} Crear una aplicación básica con una barra de navegación y definir las rutas en React Router.
    \item \textbf{IN2.2} Desarrollar la sección de inicio de sesión.
    \item \textbf{IN2.3} Desarrollar el componente de inicio.
    \item \textbf{IN2.4} Desarrollar la sección de dietas.
    \item \textbf{IN2.5} Desarrollar la sección de entrenamientos.
    \item \textbf{IN2.6} Desarrollar la logica de autenticación y manejo de sesiones.
    \item \textbf{IN2.7} Desarrollar estetica y diseño de la aplicación.
    \item \textbf{IN2.8} Desarrollar sistema de filtraje y busqueda de dietas y entrenamientos.
    \item \textbf{IN2.9} Desplegar todo en Render.
\end{itemize}

\subsubsection{Aplicación Móvil}
\begin{itemize}
    \item \textbf{IN3.1} Configurar el entorno de desarrollo con React Native CLI.
    \item \textbf{IN3.2} Crear pantallas básicas para la navegación principal.
    \item \textbf{IN3.3} Implementar la autenticación y el manejo de sesiones.
    \item \textbf{IN3.4} Desarrollar la pantalla de inicio con acceso a las funcionalidades principales.
    \item \textbf{IN3.5} Implementar la funcionalidad de seguimiento de entrenamientos en tiempo real.
    \item \textbf{IN3.6} Desarrollar la funcionalidad de registro y seguimiento de dietas diarias.
    \item \textbf{IN3.7} Probar la aplicación en dispositivos iOS y Android.
    \item \textbf{IN3.8} Exportar la aplicación a una APK.
\end{itemize}