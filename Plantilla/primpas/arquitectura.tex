La arquitectura de FitFuelBalance se ha diseñado para ser modular, escalable y mantener una separación clara entre las distintas capas del sistema. Figura \ref{FIG:arquitectura} muestra un diagrama de la arquitectura general de FitFuelBalance, que ilustra la interacción entre los distintos componentes del sistema. A continuación, se describen los principales componentes de la arquitectura y sus interacciones.

\begin{figure}[Distribución Arquitectura]{FIG:arquitectura}{Subsistemas del sistema FitFuelBalance}
    \image{12cm}{}{arquitectura}
\end{figure}

El sistema está estructurado en tres capas principales. La Capa de Presentación se divide en dos componentes: el Frontend Web y la Aplicación Móvil. El Frontend Web, desarrollado utilizando React, se encarga de la interacción con el usuario a través de una interfaz intuitiva y responsive. Para el diseño y estilo visual, se utilizan Bootstrap y CSS. Por otro lado, la Aplicación Móvil, desarrollada con React Native, permite a los usuarios acceder a los servicios de FitFuelBalance desde dispositivos móviles iOS \cite{iOS} y Android \cite{Android}. La interfaz móvil está diseñada para ser amigable y fácil de usar.

La Capa de Lógica de Negocio está representada por el Backend, desarrollado con Django. El backend maneja la lógica de negocio de la aplicación, las reglas de negocio y la gestión de datos. Además, implementa APIs RESTful \cite{RESTfulAPI} para la comunicación con el frontend y la aplicación móvil, asegurando una integración eficiente y segura entre las distintas partes del sistema.

Finalmente, la Capa de Datos se centra en la Base de Datos. Para almacenar y gestionar los datos de la aplicación se utiliza PostgreSQL. Neon gestiona esta base de datos, proporcionando alta disponibilidad y escalabilidad, lo que garantiza que los datos estén siempre accesibles y seguros, incluso a medida que la aplicación crece y se expande.

\subsubsection{Descripción de Componentes}

    El \textbf{Frontend Web} está construido utilizando \textbf{React}, lo que permite una interfaz de usuario dinámica y eficiente, facilitando la creación de componentes reutilizables y un manejo eficaz del estado de la aplicación. Para asegurar una presentación visual consistente y atractiva, se utilizan \textbf{Bootstrap} y \textbf{CSS} para estilizar la aplicación.
    
    La \textbf{Aplicación Móvil} se desarrolla con \textbf{React Native}, lo que permite el desarrollo de aplicaciones móviles multiplataforma a partir de una única base de código. React Native proporciona una experiencia de usuario nativa y acceso a funcionalidades específicas del dispositivo, asegurando una interfaz amigable y efectiva en dispositivos iOS y Android.
    
    El \textbf{Backend} se implementa utilizando \textbf{Django}, un framework de alto nivel que facilita el desarrollo rápido y limpio de aplicaciones web. Django proporciona una estructura robusta con una gran cantidad de funcionalidades integradas. Los principales módulos utilizados para controlar la lógica de la aplicación son los modelos y las vistas. La comunicación entre el frontend, la aplicación móvil y el backend se realiza a través de \textbf{APIs RESTful}, que permiten un intercambio de datos eficiente y seguro. La API se divide en tres módulos principales: usuarios, nutrición y deporte, asegurando una gestión organizada y efectiva de cada área.
    
    La \textbf{Base de Datos} utiliza \textbf{PostgreSQL}, una base de datos relacional que almacena datos estructurados de manera eficiente. La gestión de la base de datos se realiza a través de \textbf{Neon}, que proporciona capacidades avanzadas de administración y escalabilidad, garantizando que los datos estén siempre accesibles y seguros.
    

\subsubsection{Despliegue y Infraestructura}

El despliegue de FitFuelBalance se ha realizado utilizando \textbf{Render}, una plataforma que facilita la gestión y el despliegue continuo tanto del frontend como del backend. Render permite gestionar la aplicación de manera eficiente, asegurando una alta disponibilidad, escalabilidad y rendimiento, lo que permite manejar un alto volumen de usuarios y datos.
