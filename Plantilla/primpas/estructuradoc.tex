El proyecto FitFuelBalance tiene como objetivo desarrollar una plataforma integrada de nutrición y ejercicio, accesible tanto en formato web como móvil. Esta plataforma permite a los usuarios recibir planes personalizados de nutrición y ejercicio, creados por profesionales en base a sus necesidades individuales. El sistema no solo servirá como una herramienta práctica para usuarios individuales, sino también como una demostración de cómo la tecnología puede facilitar la gestión de la salud y el bienestar.

El proyecto abarca el desarrollo completo de una aplicación web y móvil, desde la fase de análisis y diseño, pasando por la implementación y pruebas, hasta su despliegue final en un entorno de producción. La plataforma está diseñada para ser accesible desde cualquier dispositivo con conexión a Internet, asegurando que los usuarios puedan interactuar con sus planes y profesionales en cualquier momento y lugar.

El ciclo de vida del proyecto está basado en un enfoque incremental, lo que permite la adición progresiva de funcionalidades y asegura que cada incremento deje el sistema en un estado funcional y robusto. Esto es especialmente beneficioso para adaptarse a nuevos requisitos y mejoras continuas basadas en la retroalimentación de los usuarios.

Con este proyecto, se busca no solo ofrecer una solución práctica a los usuarios, sino también demostrar el potencial de la integración de tecnologías modernas como Django, React, React Native y PostgreSQL en la creación de sistemas distribuidos eficientes y escalables.