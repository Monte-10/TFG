En esta sección, se examinan las soluciones y plataformas actuales en el mercado que abordan la integración de la nutrición y el ejercicio. Se identifican sus características, ventajas y limitaciones.

\subsection{Aplicaciones de Nutrición}

Existen numerosas aplicaciones que ayudan a los usuarios a gestionar su nutrición. Entre las más populares se encuentran MyFitnessPal y Yazio, cuyas características, ventajas y limitaciones se resumen en la Tabla \ref{TAB:NUTRICION}.

\begin{table}[Comparación de aplicaciones de nutrición]{TAB:NUTRICION}{Comparación de aplicaciones de nutrición}
  \begin{tabular}{|p{3cm}|p{5cm}|p{5cm}|}
    \hline
    \textbf{Aplicación} & \textbf{Ventajas} & \textbf{Limitaciones} \\
    \hline
    MyFitnessPal & \begin{itemize}
      \item Amplia base de datos de alimentos.
      \item Fácil registro de alimentos mediante escaneo de códigos de barras.
      \item Seguimiento detallado de macronutrientes.
    \end{itemize} & \begin{itemize}
      \item Planes de nutrición genéricos.
      \item Necesidad de una versión premium para acceder a funciones avanzadas.
    \end{itemize} \\
    \hline
    Yazio & \begin{itemize}
      \item Planes de nutrición personalizados.
      \item Recetas y planes de comidas saludables.
      \item Interfaz amigable y fácil de usar.
    \end{itemize} & \begin{itemize}
      \item Algunas funciones requieren suscripción premium.
      \item Menor base de datos de alimentos en comparación con MyFitnessPal.
    \end{itemize} \\
    \hline
  \end{tabular}
\end{table}

\subsection{Aplicaciones de Ejercicio}

Las aplicaciones de ejercicio proporcionan rutinas de entrenamiento y seguimiento del progreso físico. Entre las más destacadas están Nike Training Club, Fitbod y Calisteniapp, cuyas características, ventajas y limitaciones se detallan en la Tabla \ref{TAB:EJERCICIO}.

\begin{table}[Comparación de aplicaciones de ejercicio]{TAB:EJERCICIO}{Comparación de aplicaciones de ejercicio}
  \begin{tabular}{|p{3cm}|p{5cm}|p{5cm}|}
    \hline
    \textbf{Aplicación} & \textbf{Ventajas} & \textbf{Limitaciones} \\
    \hline
    Nike Training Club & \begin{itemize}
      \item Variedad de entrenamientos para diferentes niveles y objetivos.
      \item Instrucciones detalladas y videos de alta calidad.
      \item Gratis para la mayoría de las funciones.
    \end{itemize} & \begin{itemize}
      \item Falta de personalización avanzada basada en datos del usuario.
      \item Necesidad de conexión a internet para acceder a los videos.
    \end{itemize} \\
    \hline
    Fitbod & \begin{itemize}
      \item Personalización avanzada de los entrenamientos.
      \item Adaptación de ejercicios según el progreso y el equipo disponible.
      \item Seguimiento detallado del rendimiento.
    \end{itemize} & \begin{itemize}
      \item Suscripción necesaria para acceder a todas las funciones.
      \item Puede ser complejo para principiantes.
    \end{itemize} \\
    \hline
    Calisteniapp & \begin{itemize}
      \item Enfoque especializado en calistenia.
      \item Rutinas personalizables basadas en el nivel y los objetivos del usuario.
      \item Comunidad activa y funciones de seguimiento del progreso.
    \end{itemize} & \begin{itemize}
      \item Puede no ser adecuada para usuarios que prefieren entrenamientos con pesas o equipos de gimnasio.
      \item Algunas funciones avanzadas requieren suscripción premium.
    \end{itemize} \\
    \hline
  \end{tabular}
\end{table}

\subsection{Plataformas Combinadas}

Algunas plataformas intentan integrar tanto la nutrición como el ejercicio, proporcionando una solución más holística para la gestión de la salud. Ejemplos de estas plataformas son Fitbit y Samsung Health, cuyas características, ventajas y limitaciones se resumen en la Tabla \ref{TAB:COMBINADAS}.

\begin{table}[Comparación de plataformas combinadas]{TAB:COMBINADAS}{Comparación de plataformas combinadas}
  \begin{tabular}{|p{3cm}|p{5cm}|p{5cm}|}
    \hline
    \textbf{Plataforma} & \textbf{Ventajas} & \textbf{Limitaciones} \\
    \hline
    Fitbit & \begin{itemize}
      \item Integración de datos de actividad física y nutrición.
      \item Amplia gama de dispositivos wearables.
      \item Comunidad activa y funciones sociales.
    \end{itemize} & \begin{itemize}
      \item Personalización limitada en los planes de ejercicio y nutrición.
      \item Dependencia de dispositivos adicionales para obtener el máximo beneficio.
    \end{itemize} \\
    \hline
    Samsung Health & \begin{itemize}
      \item Amplia gama de funciones de salud y bienestar.
      \item Integración con dispositivos Samsung.
      \item Funciones de monitoreo del sueño y la salud en general.
    \end{itemize} & \begin{itemize}
      \item Personalización limitada en comparación con plataformas especializadas.
      \item Menor integración con dispositivos de otras marcas.
    \end{itemize} \\
    \hline
  \end{tabular}
\end{table}

\subsection{Limitaciones de las Soluciones Actuales}

A pesar de los avances, las soluciones actuales enfrentan varias limitaciones:

\begin{itemize}
  \item \textbf{Falta de Personalización}: No se consideran completamente las necesidades individuales de cada usuario.
  \item \textbf{Acceso a Asesoramiento Profesional}: La mayoría de las plataformas no proporcionan acceso directo a profesionales de la salud y el fitness, lo que limita la capacidad de obtener asesoramiento personalizado.
  \item \textbf{Costo}: Algunas aplicaciones y servicios personalizados pueden ser costosos.
  \item \textbf{Integración de Datos}: La falta de integración entre diferentes dispositivos puede dificultar una visión holística.
\end{itemize}
