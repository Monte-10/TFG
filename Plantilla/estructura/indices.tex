En esta sección, se examinan las soluciones y plataformas actuales en el mercado que abordan la integración de la nutrición y el ejercicio. Se identifican sus características, ventajas y limitaciones.

\subsection{Aplicaciones de Nutrición}

Existen numerosas aplicaciones que ayudan a los usuarios a gestionar su nutrición. Entre las más populares se encuentran MyFitnessPal \cite{MyFitnessPal} y Yazio \cite{Yazio}, cuyas características, ventajas y limitaciones se resumen en la Tabla \ref{TAB:NUTRICION}.

\begin{table}[Comparación de aplicaciones de nutrición]{TAB:NUTRICION}{Comparación de aplicaciones de nutrición}
  \begin{tabular}{|p{3cm}|p{5cm}|p{5cm}|}
    \hline
    \textbf{Aplicación} & \textbf{Ventajas} & \textbf{Limitaciones} \\
    \hline
    MyFitnessPal & MyFitnessPal ofrece una amplia base de datos de alimentos y facilita el registro de alimentos mediante escaneo de códigos de barras. También permite un seguimiento detallado de macronutrientes. & MyFitnessPal presenta planes de nutrición genéricos y requiere una versión premium para acceder a funciones avanzadas. \\
    \hline
    Yazio & Yazio proporciona planes de nutrición personalizados, recetas y planes de comidas saludables, y una interfaz amigable y fácil de usar. & Algunas funciones de Yazio requieren suscripción premium y tiene una menor base de datos de alimentos en comparación con MyFitnessPal. \\
    \hline
  \end{tabular}
\end{table}

\subsection{Aplicaciones de Ejercicio}

Las aplicaciones de ejercicio proporcionan rutinas de entrenamiento y seguimiento del progreso físico. Entre las más destacadas están Nike Training Club \cite{NikeTrainingClub}, Fitbod \cite{Fitbod} y Calisteniapp \cite{Calisteniapp}, cuyas características, ventajas y limitaciones se detallan en la Tabla \ref{TAB:EJERCICIO}.

\begin{table}[Comparación de aplicaciones de ejercicio]{TAB:EJERCICIO}{Comparación de aplicaciones de ejercicio}
  \begin{tabular}{|p{3cm}|p{5cm}|p{5cm}|}
    \hline
    \textbf{Aplicación} & \textbf{Ventajas} & \textbf{Limitaciones} \\
    \hline
    Nike Training Club & Nike Training Club ofrece una variedad de entrenamientos para diferentes niveles y objetivos, con instrucciones detalladas y videos de alta calidad, y es gratis para la mayoría de las funciones. & Carece de personalización avanzada basada en datos del usuario y requiere conexión a internet para acceder a los videos. \\
    \hline
    Fitbod & Fitbod ofrece personalización avanzada de los entrenamientos, adaptación de ejercicios según el progreso y el equipo disponible, y seguimiento detallado del rendimiento. & Necesita una suscripción para acceder a todas las funciones y puede ser complejo para principiantes. \\
    \hline
    Calisteniapp & Calisteniapp se especializa en calistenia, con rutinas personalizables basadas en el nivel y los objetivos del usuario, y una comunidad activa con funciones de seguimiento del progreso. & Puede no ser adecuada para usuarios que prefieren entrenamientos con pesas o equipos de gimnasio, y algunas funciones avanzadas requieren suscripción premium. \\
    \hline
  \end{tabular}
\end{table}

\subsection{Plataformas Combinadas}

Algunas plataformas intentan integrar tanto la nutrición como el ejercicio, proporcionando una solución más holística para la gestión de la salud. Ejemplos de estas plataformas son Fitbit \cite{Fitbit} y Samsung Health \cite{SamsungHealth}, cuyas características, ventajas y limitaciones se resumen en la Tabla \ref{TAB:COMBINADAS}.

\begin{table}[Comparación de plataformas combinadas]{TAB:COMBINADAS}{Comparación de plataformas combinadas}
  \begin{tabular}{|p{3cm}|p{5cm}|p{5cm}|}
    \hline
    \textbf{Plataforma} & \textbf{Ventajas} & \textbf{Limitaciones} \\
    \hline
    Fitbit & Fitbit integra datos de actividad física y nutrición, ofrece una amplia gama de dispositivos wearables, y cuenta con una comunidad activa y funciones sociales. & La personalización en los planes de ejercicio y nutrición es limitada y depende de dispositivos adicionales para obtener el máximo beneficio. \\
    \hline
    Samsung Health & Samsung Health ofrece una amplia gama de funciones de salud y bienestar, integración con dispositivos Samsung, y funciones de monitoreo del sueño y la salud en general. & La personalización es limitada en comparación con plataformas especializadas y tiene menor integración con dispositivos de otras marcas. \\
    \hline
  \end{tabular}
\end{table}

\subsection{Limitaciones de las Soluciones Actuales}

A pesar de los avances, las soluciones actuales enfrentan varias limitaciones. La falta de personalización no considera completamente las necesidades individuales de cada usuario. El acceso a asesoramiento profesional es limitado, ya que la mayoría de las plataformas no proporcionan acceso directo a profesionales de la salud y el fitness, lo que restringe la capacidad de obtener asesoramiento personalizado. Además, algunas aplicaciones y servicios personalizados pueden ser costosos. Finalmente, la integración de datos es insuficiente, ya que la falta de integración entre diferentes dispositivos puede dificultar una visión holística.
