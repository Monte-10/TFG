En esta sección, se explican los principales frameworks, bibliotecas y tecnologías utilizadas para el desarrollo del proyecto \textit{FitFuelBalance}.

\subsubsection{Django}

Django es un framework de alto nivel para el desarrollo web en Python que fomenta el desarrollo rápido y un diseño limpio y pragmático. Fue creado en 2005 por Adrian Holovaty y Simon Willison mientras trabajaban en el periódico \textit{Lawrence Journal-World}. Django se basa en la idea de DRY (Don’t Repeat Yourself) y ofrece herramientas poderosas como un ORM (Object-Relational Mapping), un sistema de plantillas y una interfaz administrativa automáticamente generada.

\textbf{Ventajas}:
\begin{itemize}
  \item Administración automática de la base de datos.
  \item Sistema de plantillas potente y flexible.
  \item Gran cantidad de bibliotecas y paquetes disponibles.
\end{itemize}

\textbf{Limitaciones}:
\begin{itemize}
  \item Puede ser más complejo para proyectos pequeños.
  \item Requiere familiaridad con Python y el ecosistema de Django.
\end{itemize}

\subsubsection{React}

React es una biblioteca de JavaScript para crear interfaces de usuario para sistemas distribuidos, especialmente aquellos que trabajan con datos y actualizaciones en tiempo real. Utiliza JSX, una sintaxis muy similar a HTML, lo que facilita su implementación. El sistema se basa en componentes que contienen su propia lógica y estado.

\textbf{Ventajas}:
\begin{itemize}
  \item Facilita la creación de interfaces de usuario dinámicas y responsivas.
  \item Gran comunidad y ecosistema de bibliotecas y herramientas.
  \item Modularización y reutilización de componentes.
\end{itemize}

\textbf{Limitaciones}:
\begin{itemize}
  \item Requiere un proceso de configuración inicial.
  \item La gestión del estado en aplicaciones grandes puede ser compleja sin herramientas adicionales.
\end{itemize}

\subsubsection{React Native}

React Native es un marco de desarrollo de aplicaciones móviles creado por Facebook en 2015. Permite a los desarrolladores construir aplicaciones móviles utilizando JavaScript y React, compilando a código nativo para iOS y Android.

\textbf{Ventajas}:
\begin{itemize}
  \item Reutilización de código entre plataformas iOS y Android.
  \item Ecosistema robusto y activo.
  \item Componentes nativos que garantizan un rendimiento óptimo.
\end{itemize}

\textbf{Limitaciones}:
\begin{itemize}
  \item Algunas funcionalidades nativas pueden requerir desarrollo adicional.
  \item La actualización de versiones puede requerir ajustes significativos.
\end{itemize}

\subsubsection{PostgreSQL}

PostgreSQL es un sistema de gestión de bases de datos relacional y objeto-relacional, conocido por su estabilidad, extensibilidad y cumplimiento de estándares. Es una elección popular para aplicaciones web y sistemas de información que requieren una base de datos robusta y escalable.

\textbf{Ventajas}:
\begin{itemize}
  \item Altamente extensible y soporte para tipos de datos avanzados.
  \item Transacciones ACID completas garantizando la integridad de los datos.
  \item Amplia comunidad y soporte de herramientas.
\end{itemize}

\textbf{Limitaciones}:
\begin{itemize}
  \item Puede requerir configuración y ajuste para un rendimiento óptimo.
  \item Complejidad en la administración para usuarios sin experiencia previa en bases de datos.
\end{itemize}

\subsubsection{Python}

Python es un lenguaje de programación de alto nivel, interpretado y de propósito general, conocido por su sintaxis legible y facilidad de uso. En este proyecto, Python se utiliza tanto para el desarrollo de Django como para otras funciones del backend.

\textbf{Ventajas}:
\begin{itemize}
  \item Sintaxis sencilla y clara, ideal para desarrollo rápido.
  \item Gran cantidad de bibliotecas y frameworks disponibles.
  \item Comunidad activa y extensa documentación.
\end{itemize}

\textbf{Limitaciones}:
\begin{itemize}
  \item Puede no ser tan rápido como lenguajes compilados en ciertas tareas.
  \item Requiere un buen manejo de entornos virtuales para la gestión de dependencias.
\end{itemize}

\subsubsection{JavaScript}

JavaScript es un lenguaje de programación interpretado que se utiliza principalmente para el desarrollo de aplicaciones web dinámicas. Es el lenguaje más común para el desarrollo frontend y se utiliza en este proyecto junto con React.

\textbf{Ventajas}:
\begin{itemize}
  \item Ampliamente soportado por todos los navegadores web.
  \item Gran cantidad de frameworks y bibliotecas para desarrollo web.
  \item Ideal para la creación de aplicaciones web interactivas y dinámicas.
\end{itemize}

\textbf{Limitaciones}:
\begin{itemize}
  \item Problemas de compatibilidad entre diferentes navegadores.
  \item Requiere una gestión adecuada del código para evitar problemas de rendimiento.
\end{itemize}

\subsubsection{Bootstrap y CSS}

Bootstrap es un framework de front-end de código abierto que permite diseñar sitios y aplicaciones web de forma rápida y sencilla. CSS (Cascading Style Sheets) es el lenguaje utilizado para describir la presentación de un documento escrito en HTML o XML.

\textbf{Ventajas}:
\begin{itemize}
  \item Facilita la creación de interfaces responsivas y atractivas.
  \item Gran cantidad de componentes predefinidos.
  \item Compatible con todos los navegadores modernos.
\end{itemize}

\textbf{Limitaciones}:
\begin{itemize}
  \item Puede resultar en sitios web que se ven similares si no se personaliza adecuadamente.
  \item Dependencia de archivos externos puede afectar el rendimiento si no se gestiona bien.
\end{itemize}

Estas tecnologías han sido seleccionadas y utilizadas en el proyecto \textit{FitFuelBalance} para asegurar un desarrollo eficiente, una experiencia de usuario óptima y una alta escalabilidad del sistema.