En esta sección, se explican los principales frameworks, bibliotecas y tecnologías utilizadas para el desarrollo del proyecto \textit{FitFuelBalance}.

\subsubsection{Django}

Django es un framework de alto nivel para el desarrollo web en Python que fomenta el desarrollo rápido y un diseño limpio y pragmático. Fue creado en 2005 por Adrian Holovaty y Simon Willison mientras trabajaban en el periódico \textit{Lawrence Journal-World}. Django se basa en la idea de DRY (Don’t Repeat Yourself) y ofrece herramientas poderosas como un ORM (Object-Relational Mapping), un sistema de plantillas y una interfaz administrativa automáticamente generada. Las ventajas y limitaciones se resumen en la Tabla \ref{TAB:Django}.

\begin{table}[Django]{TAB:Django}{Ventajas y limitaciones de Django}
  \begin{tabular}{|p{7cm}|p{7cm}|}
    \hline
    \textbf{Ventajas} & \textbf{Limitaciones} \\
    \hline
    Django permite la administración automática de la base de datos, tiene un sistema de plantillas potente y flexible y ofrece una gran cantidad de bibliotecas y paquetes disponibles. & Puede ser más complejo para proyectos pequeños y requiere familiaridad con Python y el ecosistema de Django. \\
    \hline
  \end{tabular}
\end{table}

\subsubsection{React}

React es una biblioteca de JavaScript para crear interfaces de usuario para sistemas distribuidos, especialmente aquellos que trabajan con datos y actualizaciones en tiempo real. Utiliza JSX, una sintaxis muy similar a HTML, lo que facilita su implementación. El sistema se basa en componentes que contienen su propia lógica y estado. Las ventajas y limitaciones se resumen en la Tabla \ref{TAB:React}.

\begin{table}[React]{TAB:React}{Ventajas y limitaciones de React}
  \begin{tabular}{|p{7cm}|p{7cm}|}
    \hline
    \textbf{Ventajas} & \textbf{Limitaciones} \\
    \hline
    React facilita la creación de interfaces de usuario dinámicas y responsivas, cuenta con una gran comunidad y un ecosistema de bibliotecas y herramientas, y permite la modularización y reutilización de componentes. & Requiere un proceso de configuración inicial y la gestión del estado en aplicaciones grandes puede ser compleja sin herramientas adicionales. \\
    \hline
  \end{tabular}
\end{table}

\subsubsection{React Native}

React Native es un marco de desarrollo de aplicaciones móviles creado por Facebook en 2015. Permite a los desarrolladores construir aplicaciones móviles utilizando JavaScript y React, compilando a código nativo para iOS y Android. Las ventajas y limitaciones se resumen en la Tabla \ref{TAB:React_Native}.

\begin{table}[React Native]{TAB:React_Native}{Ventajas y limitaciones de React Native}
  \begin{tabular}{|p{7cm}|p{7cm}|}
    \hline
    \textbf{Ventajas} & \textbf{Limitaciones} \\
    \hline
    React Native permite la reutilización de código entre plataformas iOS y Android, cuenta con un ecosistema robusto y activo, y ofrece componentes nativos que garantizan un rendimiento óptimo. & Algunas funcionalidades nativas pueden requerir desarrollo adicional y la actualización de versiones puede requerir ajustes significativos. \\
    \hline
  \end{tabular}
\end{table}

\subsubsection{PostgreSQL}

PostgreSQL es un sistema de gestión de bases de datos relacional y objeto-relacional, conocido por su estabilidad, extensibilidad y cumplimiento de estándares. Es una elección popular para aplicaciones web y sistemas de información que requieren una base de datos robusta y escalable. Las ventajas y limitaciones se resumen en la Tabla \ref{TAB:PostgreSQL}.

\begin{table}[PostgreSQL]{TAB:PostgreSQL}{Ventajas y limitaciones de PostgreSQL}
  \begin{tabular}{|p{7cm}|p{7cm}|}
    \hline
    \textbf{Ventajas} & \textbf{Limitaciones} \\
    \hline
    PostgreSQL es altamente extensible y ofrece soporte para tipos de datos avanzados, garantiza transacciones ACID completas para la integridad de los datos, y cuenta con una amplia comunidad y soporte de herramientas. & Puede requerir configuración y ajuste para un rendimiento óptimo y la administración puede ser compleja para usuarios sin experiencia previa en bases de datos. \\
    \hline
  \end{tabular}
\end{table}

\subsubsection{Python}

Python es un lenguaje de programación de alto nivel, interpretado y de propósito general, conocido por su sintaxis legible y facilidad de uso. En este proyecto, Python se utiliza tanto para el desarrollo de Django como para otras funciones del backend. Las ventajas y limitaciones se resumen en la Tabla \ref{TAB:Python}.

\begin{table}[Python]{TAB:Python}{Ventajas y limitaciones de Python}
  \begin{tabular}{|p{7cm}|p{7cm}|}
    \hline
    \textbf{Ventajas} & \textbf{Limitaciones} \\
    \hline
    Python tiene una sintaxis sencilla y clara, ideal para desarrollo rápido, ofrece una gran cantidad de bibliotecas y frameworks disponibles, y cuenta con una comunidad activa y extensa documentación. & Puede no ser tan rápido como lenguajes compilados en ciertas tareas y requiere un buen manejo de entornos virtuales para la gestión de dependencias. \\
    \hline
  \end{tabular}
\end{table}

\subsubsection{JavaScript}

JavaScript es un lenguaje de programación interpretado que se utiliza principalmente para el desarrollo de aplicaciones web dinámicas. Es el lenguaje más común para el desarrollo frontend y se utiliza en este proyecto junto con React. Las ventajas y limitaciones se resumen en la Tabla \ref{TAB:JavaScript}.

\begin{table}[JavaScript]{TAB:JavaScript}{Ventajas y limitaciones de JavaScript}
  \begin{tabular}{|p{7cm}|p{7cm}|}
    \hline
    \textbf{Ventajas} & \textbf{Limitaciones} \\
    \hline
    JavaScript es ampliamente soportado por todos los navegadores web, cuenta con una gran cantidad de frameworks y bibliotecas para desarrollo web, y es ideal para la creación de aplicaciones web interactivas y dinámicas. & Puede tener problemas de compatibilidad entre diferentes navegadores y requiere una gestión adecuada del código para evitar problemas de rendimiento. \\
    \hline
  \end{tabular}
\end{table}

\subsubsection{Bootstrap y CSS}

Bootstrap es un framework de front-end de código abierto que permite diseñar sitios y aplicaciones web de forma rápida y sencilla. CSS (Cascading Style Sheets) es el lenguaje utilizado para describir la presentación de un documento escrito en HTML o XML. Las ventajas y limitaciones se resumen en la Tabla \ref{TAB:Bootstrap_CSS}.

\begin{table}[Bootstrap y CSS]{TAB:Bootstrap_CSS}{Ventajas y limitaciones de Bootstrap y CSS}
  \begin{tabular}{|p{7cm}|p{7cm}|}
    \hline
    \textbf{Ventajas} & \textbf{Limitaciones} \\
    \hline
    Bootstrap y CSS facilitan la creación de interfaces responsivas y atractivas, ofrecen una gran cantidad de componentes predefinidos, y son compatibles con todos los navegadores modernos. & Pueden resultar en sitios web que se ven similares si no se personaliza adecuadamente y la dependencia de archivos externos puede afectar el rendimiento si no se gestiona bien. \\
    \hline
  \end{tabular}
\end{table}