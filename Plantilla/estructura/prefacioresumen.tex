En esta sección, se explican los principales frameworks, bibliotecas y tecnologías utilizadas para el desarrollo del proyecto \textit{FitFuelBalance}.

\subsubsection{Django}

Django es un framework de alto nivel para el desarrollo web en Python que fomenta el desarrollo rápido y un diseño limpio y pragmático. Fue creado en 2005 por Adrian Holovaty y Simon Willison mientras trabajaban en el periódico \textit{Lawrence Journal-World}. Django se basa en la idea de DRY (Don’t Repeat Yourself) y ofrece herramientas poderosas como un ORM (Object-Relational Mapping), un sistema de plantillas y una interfaz administrativa automáticamente generada\cite{Django}.

Las ventajas de Django incluyen la administración automática de la base de datos, un sistema de plantillas potente y flexible, y una gran cantidad de bibliotecas y paquetes disponibles. Sin embargo, puede ser más complejo para proyectos pequeños y requiere conocimiento en Python.

\subsubsection{React}

React es una biblioteca de JavaScript para crear interfaces de usuario para sistemas distribuidos, especialmente aquellos que trabajan con datos y actualizaciones en tiempo real. Utiliza JSX, una sintaxis muy similar a HTML, lo que facilita su implementación. El sistema se basa en componentes que contienen su propia lógica y estado\cite{React}.

Entre las ventajas de React se encuentran la facilidad para crear interfaces de usuario dinámicas y responsivas, una gran comunidad y un ecosistema de bibliotecas y herramientas, así como la modularización y reutilización de componentes. Las limitaciones incluyen un proceso de configuración inicial y la complejidad en la gestión del estado en aplicaciones grandes sin herramientas adicionales.

\subsubsection{React Native}

React Native es un framework de aplicaciones móviles creado por Meta en 2015. Permite a los desarrolladores construir aplicaciones móviles utilizando JavaScript y React, compilando a código nativo para iOS y Android\cite{ReactNative}.

Las ventajas de React Native incluyen la reutilización de código entre plataformas iOS y Android, un ecosistema robusto y activo, y componentes nativos que garantizan un rendimiento óptimo. No obstante, algunas funcionalidades nativas pueden requerir desarrollo adicional y la actualización de versiones puede requerir ajustes significativos.

\subsubsection{PostgreSQL}

PostgreSQL \cite{PostgreSQL} es un sistema de gestión de bases de datos relacional y objeto-relacional, conocido por su estabilidad, extensibilidad y cumplimiento de estándares. Es una elección popular para aplicaciones web y sistemas de información que requieren una base de datos robusta y escalable.

Las ventajas de PostgreSQL incluyen su alta extensibilidad y soporte para tipos de datos avanzados. Sin embargo, puede requerir configuración y ajuste para un rendimiento óptimo.

\subsubsection{Python}

Python es un lenguaje de programación de alto nivel, interpretado y de propósito general, conocido por su sintaxis legible y facilidad de uso. En este proyecto, Python se utiliza tanto para el desarrollo de Django como para otras funciones del backend\cite{Python}.

Las ventajas de Python son su sintaxis sencilla y clara, ideal para desarrollo rápido, una gran cantidad de bibliotecas y frameworks disponibles, y una comunidad activa y extensa documentación. No obstante, puede no ser tan rápido como lenguajes compilados en ciertas tareas y requiere un buen manejo de entornos virtuales para la gestión de dependencias.

\subsubsection{JavaScript}

JavaScript es un lenguaje de programación interpretado que se utiliza principalmente para el desarrollo de aplicaciones web dinámicas. Es el lenguaje más común para el desarrollo frontend y se utiliza en este proyecto junto con React\cite{JavaScript}.

Las ventajas de JavaScript incluyen su amplio soporte por todos los navegadores web, una gran cantidad de frameworks y bibliotecas para desarrollo web, y su idoneidad para la creación de aplicaciones web interactivas y dinámicas. Las limitaciones incluyen problemas de compatibilidad entre diferentes navegadores y la necesidad de una gestión adecuada del código para evitar problemas de rendimiento.

\subsubsection{Bootstrap y CSS}

Bootstrap es un framework de front-end de código abierto que permite diseñar sitios y aplicaciones web de forma rápida y sencilla. CSS (Cascading Style Sheets) es el lenguaje utilizado para describir la presentación de un documento escrito en HTML o XML\cite{Bootstrap,CSS}.

Las ventajas de Bootstrap y CSS incluyen la facilidad para crear interfaces responsivas y atractivas, una gran cantidad de componentes predefinidos, y la compatibilidad los navegadores modernos. Sin embargo, pueden resultar en sitios web que se ven similares si no se personaliza adecuadamente y la dependencia de archivos externos puede afectar el rendimiento si no se gestiona bien.