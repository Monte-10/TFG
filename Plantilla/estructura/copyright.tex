En esta sección, se examinan las soluciones y plataformas actuales en el mercado que abordan la integración de la nutrición y el ejercicio. Se identifican sus características, ventajas y limitaciones.

\subsection{Aplicaciones de Nutrición Personalizada}

Existen diversas plataformas que ofrecen planes de nutrición personalizados adaptados a las necesidades individuales de los usuarios. Las características, ventajas y limitaciones de algunas de estas plataformas se resumen en la Tabla \ref{TAB:NUTRICION_PERSONALIZADA}.

\begin{table}[Plataformas de Nutrición Personalizada]{TAB:NUTRICION_PERSONALIZADA}{Comparación de plataformas de nutrición personalizada}
  \begin{tabular}{|p{3cm}|p{5cm}|p{5cm}|}
    \hline
    \textbf{Plataforma} & \textbf{Ventajas} & \textbf{Limitaciones} \\
    \hline
    Nutrifix & \begin{itemize}
      \item Personalización basada en datos de actividad física.
      \item Consideración de restricciones dietéticas y preferencias.
    \end{itemize} & \begin{itemize}
      \item Necesidad de sincronizar con otras aplicaciones para obtener datos precisos.
      \item Algunas funciones avanzadas requieren una suscripción.
    \end{itemize} \\
    \hline
    PlateJoy & \begin{itemize}
      \item Listas de compras y recetas personalizadas.
      \item Adaptación a diversas necesidades dietéticas (vegetariano, sin gluten, etc.).
    \end{itemize} & \begin{itemize}
      \item Suscripción necesaria para acceder a todas las funciones.
      \item Dependencia de la precisión de los datos ingresados por el usuario.
    \end{itemize} \\
    \hline
  \end{tabular}
\end{table}

\subsection{Aplicaciones de Ejercicio Personalizado}

Las aplicaciones de ejercicio proporcionan rutinas de entrenamiento y seguimiento del progreso físico. Las características, ventajas y limitaciones de algunas de estas aplicaciones se detallan en la Tabla \ref{TAB:EJERCICIO_PERSONALIZADO}.

\begin{table}[Plataformas de Ejercicio Personalizado]{TAB:EJERCICIO_PERSONALIZADO}{Comparación de plataformas de ejercicio personalizado}
  \begin{tabular}{|p{3cm}|p{5cm}|p{5cm}|}
    \hline
    \textbf{Plataforma} & \textbf{Ventajas} & \textbf{Limitaciones} \\
    \hline
    Freeletics & \begin{itemize}
      \item Planes de entrenamiento personalizados y adaptativos.
      \item Videos instructivos y seguimiento del rendimiento.
    \end{itemize} & \begin{itemize}
      \item Suscripción necesaria para acceder a funciones premium.
      \item Puede no ser adecuado para usuarios que prefieren entrenamiento con equipos específicos.
    \end{itemize} \\
    \hline
    JEFIT & \begin{itemize}
      \item Base de datos extensa de ejercicios.
      \item Comunidad activa para compartir y comparar rutinas.
    \end{itemize} & \begin{itemize}
      \item Requiere suscripción para acceder a algunas funciones avanzadas.
      \item Personalización limitada sin la versión premium.
    \end{itemize} \\
    \hline
  \end{tabular}
\end{table}

\subsection{Plataformas Combinadas}

Algunas plataformas buscan integrar tanto la nutrición como el ejercicio para ofrecer una solución completa. Las características, ventajas y limitaciones de algunas de estas plataformas se resumen en la Tabla \ref{TAB:COMBINADAS_PERSONALIZADAS}.

\begin{table}[Plataformas Combinadas]{TAB:COMBINADAS_PERSONALIZADAS}{Comparación de plataformas combinadas}
  \begin{tabular}{|p{3cm}|p{5cm}|p{5cm}|}
    \hline
    \textbf{Plataforma} & \textbf{Ventajas} & \textbf{Limitaciones} \\
    \hline
    8fit & \begin{itemize}
      \item Integración de planes de nutrición y ejercicio.
      \item Videos instructivos y listas de compras.
    \end{itemize} & \begin{itemize}
      \item Suscripción necesaria para acceder a todas las funciones.
      \item Requiere consistencia en el ingreso de datos por parte del usuario.
    \end{itemize} \\
    \hline
    Noom & \begin{itemize}
      \item Enfoque en la psicología del comportamiento para cambios duraderos.
      \item Integración de dieta y ejercicio en un solo plan.
    \end{itemize} & \begin{itemize}
      \item Suscripción necesaria para acceder a las funciones completas.
      \item Puede ser más caro en comparación con otras aplicaciones.
    \end{itemize} \\
    \hline
  \end{tabular}
\end{table}

\subsection{Limitaciones y Oportunidades de Mejora}

A pesar de los avances en la personalización de planes de nutrición y ejercicio, las soluciones actuales todavía enfrentan desafíos. Las principales limitaciones y oportunidades de mejora se resumen en la Tabla \ref{TAB:LIMITACIONES}.

\begin{table}[Limitaciones y Oportunidades de Mejora]{TAB:LIMITACIONES}{Limitaciones y oportunidades de mejora}
  \begin{tabular}{|p{5cm}|p{8cm}|}
    \hline
    \textbf{Limitación} & \textbf{Descripción} \\
    \hline
    Falta de Personalización & Muchas aplicaciones ofrecen planes genéricos que no consideran completamente las necesidades individuales de cada usuario. \\
    \hline
    Acceso a Asesoramiento Profesional & La mayoría de las plataformas no proporcionan acceso directo a profesionales de la salud y el fitness, lo que limita la capacidad de obtener asesoramiento personalizado. \\
    \hline
    Costo & Algunas aplicaciones y servicios personalizados pueden ser costosos, lo que restringe el acceso para muchos usuarios. \\
    \hline
    Integración de Datos & La falta de integración entre diferentes dispositivos y aplicaciones puede dificultar una visión holística de la salud del usuario. \\
    \hline
  \end{tabular}
\end{table}

Estas limitaciones abren oportunidades para el desarrollo de nuevas plataformas que puedan ofrecer un mayor nivel de personalización, accesibilidad y apoyo profesional.
