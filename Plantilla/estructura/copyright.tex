En esta sección, se examinan las soluciones y plataformas actuales en el mercado que abordan la integración de la nutrición y el ejercicio pero que permiten una personalización según el usuario y sus características. Se identifican sus características, ventajas y limitaciones.

\subsection{Aplicaciones de Nutrición Personalizada}

Existen diversas plataformas que ofrecen planes de nutrición personalizados adaptados a las necesidades individuales de los usuarios. Las características, ventajas y limitaciones de algunas de estas plataformas se resumen en la Tabla \ref{TAB
}.

\begin{table}[Plataformas de Nutrición Personalizada]{TAB
}{Comparación de plataformas de nutrición personalizada}
\begin{tabular}{|p{3cm}|p{5cm}|p{5cm}|}
\hline
\textbf{Plataforma} & \textbf{Ventajas} & \textbf{Limitaciones} \
\hline
Nutrifix & Nutrifix ofrece personalización basada en datos de actividad física y considera restricciones dietéticas y preferencias. & Sin embargo, necesita sincronizar con otras aplicaciones para obtener datos precisos y algunas funciones avanzadas requieren una suscripción. \
\hline
PlateJoy & PlateJoy proporciona listas de compras y recetas personalizadas, además de adaptarse a diversas necesidades dietéticas (vegetariano, sin gluten, etc.). & No obstante, requiere una suscripción para acceder a todas las funciones y depende de la precisión de los datos ingresados por el usuario. \
\hline
\end{tabular}
\end{table}

\subsection{Aplicaciones de Ejercicio Personalizado}

Las aplicaciones de ejercicio proporcionan rutinas de entrenamiento y seguimiento del progreso físico. Las características, ventajas y limitaciones de algunas de estas aplicaciones se detallan en la Tabla \ref{TAB
}.

\begin{table}[Plataformas de Ejercicio Personalizado]{TAB
}{Comparación de plataformas de ejercicio personalizado}
\begin{tabular}{|p{3cm}|p{5cm}|p{5cm}|}
\hline
\textbf{Plataforma} & \textbf{Ventajas} & \textbf{Limitaciones} \
\hline
Freeletics & Freeletics ofrece planes de entrenamiento personalizados y adaptativos, además de videos instructivos y seguimiento del rendimiento. & Sin embargo, requiere una suscripción para acceder a funciones premium y puede no ser adecuado para usuarios que prefieren entrenamiento con equipos específicos. \
\hline
JEFIT & JEFIT cuenta con una base de datos extensa de ejercicios y una comunidad activa para compartir y comparar rutinas. & No obstante, requiere suscripción para acceder a algunas funciones avanzadas y la personalización es limitada sin la versión premium. \
\hline
\end{tabular}
\end{table}

\subsection{Plataformas Combinadas}

Algunas plataformas buscan integrar tanto la nutrición como el ejercicio para ofrecer una solución completa. Las características, ventajas y limitaciones de algunas de estas plataformas se resumen en la Tabla \ref{TAB
}.

\begin{table}[Plataformas Combinadas]{TAB
}{Comparación de plataformas combinadas}
\begin{tabular}{|p{3cm}|p{5cm}|p{5cm}|}
\hline
\textbf{Plataforma} & \textbf{Ventajas} & \textbf{Limitaciones} \
\hline
8fit & 8fit integra planes de nutrición y ejercicio, ofrece videos instructivos y listas de compras. & Sin embargo, requiere una suscripción para acceder a todas las funciones y necesita consistencia en el ingreso de datos por parte del usuario. \
\hline
Noom & Noom se enfoca en la psicología del comportamiento para cambios duraderos e integra dieta y ejercicio en un solo plan. & No obstante, requiere suscripción para acceder a las funciones completas y puede ser más caro en comparación con otras aplicaciones. \
\hline
\end{tabular}
\end{table}

\subsection{Limitaciones y Oportunidades de Mejora}

A pesar de los avances en la personalización de planes de nutrición y ejercicio, las soluciones actuales todavía enfrentan desafíos. Las principales limitaciones y oportunidades de mejora se resumen en la Tabla \ref{TAB
}.

\begin{table}[Limitaciones y Oportunidades de Mejora]{TAB
}{Limitaciones y oportunidades de mejora}
\begin{tabular}{|p{5cm}|p{8cm}|}
\hline
\textbf{Limitación} & \textbf{Descripción} \
\hline
Falta de Personalización & Muchas aplicaciones ofrecen planes genéricos que no consideran completamente las necesidades individuales de cada usuario. \
\hline
Acceso a Asesoramiento Profesional & La mayoría de las plataformas no proporcionan acceso directo a profesionales de la salud y el fitness, lo que limita la capacidad de obtener asesoramiento personalizado. \
\hline
Costo & Algunas aplicaciones y servicios personalizados pueden ser costosos, lo que restringe el acceso para muchos usuarios. \
\hline
Integración de Datos & La falta de integración entre diferentes dispositivos y aplicaciones puede dificultar una visión holística de la salud del usuario. \
\hline
\end{tabular}
\end{table}

Estas limitaciones abren oportunidades para el desarrollo de nuevas plataformas que puedan ofrecer un mayor nivel de personalización, accesibilidad y apoyo profesional.