La siguiente sección detalla las pruebas realizadas en el proyecto FitFuelBalance, incluyendo pruebas del back-end y pruebas del front-end. Todas las pruebas han sido realizadas en entornos controlados aparte del entorno de producción, para asegurar que el sistema funcione correctamente y sin errores. También aclarar que para la realización de las pruebas se ha desactivado el uso tokens de autenticación para facilitar el acceso a los endpoints.
\section{Pruebas del Back-End}

Las pruebas del back-end se han centrado en verificar la funcionalidad de la API y asegurar que todas las operaciones del servidor se realicen correctamente. Para esto, se han utilizado herramientas como \textit{pytest} para ejecutar pruebas automatizadas y \textit{Postman} para realizar pruebas manuales de los distintos endpoints de la API. Se ha creado un testsettings.py para configurar las pruebas de Django. Todas las pruebas se encuentran en la sección \ref{SEC:PRUEBASBACKEND} del Apéndice \ref{CAP:CODIGO}.

\subsubsection{Configuración de Pruebas}
Para realizar las pruebas, se configuró un entorno de pruebas separado en el que se llevaron a cabo pruebas. Como se ha comentado anteriormente, para no causar problemas con el proyecto en producción todas las pruebas han sido realizadas en un entorno y una base de datos separada. Se ha creado un archivo de configuración de pruebas testsettings.py para configurar las pruebas de Django. En este archivo se han configurado las credenciales de la base de datos de pruebas y se han desactivado las migraciones automáticas para evitar conflictos con la base de datos de producción. La configuración de este archivo se muestra en la figura \ref{FIG:testsettings}.
La configuración de los tests de usuario se representan en la figura \ref{FIG:usertestsconfig}. La de deporte en la figura \ref{FIG:sporttestsconfig} y la de nutrición en la figura \ref{FIG:nutritiontestsconfig}.

\subsubsection{Pruebas de Endpoints}
Se han realizado pruebas a todos los endpoints críticos del back-end, asegurando que cada uno de ellos responda adecuadamente y maneje los errores de manera eficiente. Todas las aplicaciones del proyecto han sido probadas, incluyendo las aplicaciones de usuario, deporte y nutrición. La figura \ref{FIG:usertests} muestra un ejemplo de pruebas realizadas a la aplicación de usuario. La figura \ref{FIG:sporttests} muestra un ejemplo de pruebas realizadas a la aplicación de deporte y la figura \ref{FIG:nutritiontests} muestra un ejemplo de pruebas realizadas a la aplicación de nutrición.
\section{Pruebas del Front-End}

Las pruebas del front-end son esenciales para asegurar que la interfaz de usuario sea intuitiva, funcional y libre de errores. Para esto, se han utilizado diferentes técnicas y herramientas para probar tanto la aplicación web como la aplicación móvil.

\subsubsection{Pruebas de la Aplicación Web}
Para la aplicación web, se ha utilizado \textit{Jest} y \textit{React Testing Library} para realizar las pruebas de sus componentes. Estas herramientas permiten simular eventos y verificar que los componentes reaccionen adecuadamente a diferentes interacciones del usuario.

\paragraph{Prueba de Componentes}
Se ha realizado una serie de pruebas a los componentes principales de la aplicación web para asegurarse de que se rendericen correctamente y manejen los eventos de usuario como se espera.  Los tests se encuentran en la sección \ref{SEC:PRUEBASFRONTEND} del Apéndice \ref{CAP:CODIGO}.
\subsubsection{Pruebas de la Aplicación Móvil}
Para la aplicación móvil desarrollada con \textit{React Native}, se han utilizado herramientas como \textit{Jest} y \textit{React Native Testing Library} para realizar pruebas unitarias. Además, se ha utilizado \textit{React Native CLI} para probar la aplicación en diferentes dispositivos y asegurar la compatibilidad. 

\paragraph{Prueba de Componentes Móviles}
Las pruebas de componentes móviles aseguran que los elementos de la interfaz de usuario se rendericen correctamente en diferentes tamaños de pantalla y manejen las interacciones del usuario de manera adecuada. Las pruebas se encuentran en la subsección \ref{SEC:PRUEBASMOVIL} de la sección \ref{SEC:PRUEBASFRONTEND} del Apéndice \ref{CAP:CODIGO}.

\section{Implementación}

La implementación de la plataforma ha sido un proceso que ha involucrado varias fases, desde el desarrollo inicial hasta el despliegue en un entorno de producción. Esta sección describe las principales etapas del proceso de implementación, incluyendo la configuración de los servidores, el despliegue de la base de datos y las aplicaciones, y la integración continua.

\subsubsection{Despliegue del Back-End}
El back-end de FitFuelBalance, desarrollado en Django, ha sido desplegado en la plataforma Render, que proporciona un entorno escalable y fácil de gestionar. A continuación, se describen los pasos principales seguidos durante el despliegue:

\begin{itemize}
    \item \textbf{Configuración del Entorno:} Se configuraron las variables de entorno necesarias, incluyendo las credenciales de la base de datos y las claves secretas de la aplicación.
    \item \textbf{Despliegue del Código:} Se utilizó Git para manejar el control de versiones y despliegue continuo. El código fue desplegado desde el repositorio de GitHub a Render.
    \item \textbf{Migraciones de la Base de Datos:} Se ejecutaron las migraciones de Django para configurar la estructura de la base de datos en PostgreSQL.
    \item \textbf{Pruebas y Verificación:} Se realizaron pruebas post-despliegue para asegurar que el servidor estuviera funcionando correctamente y todas las API estuvieran accesibles.
\end{itemize}

\subsubsection{Despliegue del Front-End}
El front-end de la plataforma, desarrollado en React, fue desplegado en la plataforma Render de nuevo. Los pasos seguidos incluyen:

\begin{itemize}
    \item \textbf{Configuración del Proyecto:} Se configuraron las variables de entorno necesarias para la comunicación con el back-end.
    \item \textbf{Despliegue del Código:} Similar al back-end, se utilizó GitHub para manejar el despliegue continuo. Cada push al repositorio de producción desencadena un nuevo despliegue en Render.
    \item \textbf{Optimización y Compilación:} Se realizaron optimizaciones de compilación para mejorar el rendimiento y reducir el tiempo de carga de la aplicación.
    \item \textbf{Pruebas y Verificación:} Se verificó que la aplicación web estuviera funcionando correctamente y se realizaron pruebas de rendimiento.
\end{itemize}