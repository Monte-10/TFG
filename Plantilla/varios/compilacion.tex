Para compilar el documento se puede hacer de múltiples formas y aquí se van a contar dos posibles formas. Una compilación manual basada en pdflatex y otra basada en arara.

\section{Pruebas del Back-End}

Las pruebas del back-end se han centrado en verificar la funcionalidad de la API y asegurar que todas las operaciones del servidor se realicen correctamente. Para esto, se han utilizado herramientas como \textit{pytest} para ejecutar pruebas automatizadas y \textit{Postman} para realizar pruebas manuales de los distintos endpoints de la API.

\subsubsection{Configuración de Pruebas}
Para realizar las pruebas, se configuró un entorno de pruebas separado en el que se llevaron a cabo las siguientes acciones:

\begin{verbatim}
***FUTURO CODIGO DE TEST***
\end{verbatim}

En este ejemplo, se prueba el registro de un usuario en la API, verificando que la respuesta del servidor sea correcta y que el usuario se cree en la base de datos.

\subsubsection{Pruebas de Endpoints}
Se han realizado pruebas a todos los endpoints críticos del back-end, asegurando que cada uno de ellos responda adecuadamente y maneje los errores de manera eficiente. A continuación, se muestra un ejemplo de prueba de un endpoint de obtención de ejercicios:

\begin{verbatim}
***FUTURO CODIGO DE TEST***
\end{verbatim}

\section{Pruebas del Front-End}

Las pruebas del front-end son esenciales para asegurar que la interfaz de usuario sea intuitiva, funcional y libre de errores. Para esto, se han utilizado diferentes técnicas y herramientas para probar tanto la aplicación web como la aplicación móvil.

\subsubsection{Pruebas de la Aplicación Web}
Para la aplicación web, se ha utilizado \textit{Jest} y \textit{React Testing Library} para realizar pruebas unitarias y de integración. Estas herramientas permiten simular eventos y verificar que los componentes reaccionen adecuadamente a diferentes interacciones del usuario.

\paragraph{Prueba de Componentes}
Se ha realizado una serie de pruebas a los componentes principales de la aplicación web para asegurarse de que se rendericen correctamente y manejen los eventos de usuario como se espera. A continuación, se muestra un ejemplo de prueba de un componente de formulario:

\begin{verbatim}
***FUTURO CODIGO DE TEST***
\end{verbatim}

En este ejemplo, se prueba que el componente de formulario de login maneje adecuadamente la entrada del usuario y muestre los mensajes de error correctos si los campos están vacíos.

\paragraph{Pruebas de Integración}
Además de las pruebas unitarias, se han realizado pruebas de integración para verificar que diferentes partes de la aplicación funcionen juntas sin problemas. A continuación, se muestra un ejemplo de prueba de integración para verificar el flujo de login:

\begin{verbatim}
***FUTURO CODIGO DE TEST***
\end{verbatim}

\subsubsection{Pruebas de la Aplicación Móvil}
Para la aplicación móvil desarrollada con \textit{React Native}, se han utilizado herramientas como \textit{Jest} y \textit{React Native Testing Library} para realizar pruebas unitarias. Además, se ha utilizado \textit{Expo} para probar la aplicación en diferentes dispositivos y asegurar la compatibilidad.

\paragraph{Prueba de Componentes Móviles}
Las pruebas de componentes móviles aseguran que los elementos de la interfaz de usuario se rendericen correctamente en diferentes tamaños de pantalla y manejen las interacciones del usuario de manera adecuada. A continuación, se muestra un ejemplo de prueba de un componente de botón en la aplicación móvil:

\begin{verbatim}
***FUTURO CODIGO DE TEST***
\end{verbatim}

\paragraph{Pruebas de Integración Móviles}
Se han realizado pruebas de integración en la aplicación móvil para asegurar que la navegación entre pantallas y la comunicación con el backend sean fluidas y sin errores. A continuación, se muestra un ejemplo de prueba de integración que verifica el flujo de registro de usuario en la aplicación móvil:

\begin{verbatim}
***FUTURO CODIGO DE TEST***
\end{verbatim}

\section{Implementación}

La implementación de la plataforma FitFuelBalance ha sido un proceso detallado y meticuloso que ha involucrado varias fases, desde el desarrollo inicial hasta el despliegue en un entorno de producción. Esta sección describe las principales etapas del proceso de implementación, incluyendo la configuración de los servidores, el despliegue de la base de datos y las aplicaciones, y la integración continua.

\subsubsection{Despliegue del Back-End}
El back-end de FitFuelBalance, desarrollado en Django, ha sido desplegado en la plataforma Render, que proporciona un entorno escalable y fácil de gestionar. A continuación, se describen los pasos principales seguidos durante el despliegue:

\begin{itemize}
    \item \textbf{Configuración del Entorno:} Se configuraron las variables de entorno necesarias, incluyendo las credenciales de la base de datos y las claves secretas de la aplicación.
    \item \textbf{Despliegue del Código:} Se utilizó Git para manejar el control de versiones y despliegue continuo. El código fue desplegado desde el repositorio de GitHub a Render.
    \item \textbf{Migraciones de la Base de Datos:} Se ejecutaron las migraciones de Django para configurar la estructura de la base de datos en PostgreSQL.
    \item \textbf{Pruebas y Verificación:} Se realizaron pruebas post-despliegue para asegurar que el servidor estuviera funcionando correctamente y todas las API estuvieran accesibles.
\end{itemize}

\subsubsection{Despliegue del Front-End}
El front-end de la plataforma, desarrollado en React, fue desplegado en la plataforma Render de nuevo. Los pasos seguidos incluyen:

\begin{itemize}
    \item \textbf{Configuración del Proyecto:} Se configuraron las variables de entorno necesarias para la comunicación con el back-end.
    \item \textbf{Despliegue del Código:} Similar al back-end, se utilizó GitHub para manejar el despliegue continuo. Cada push al repositorio de producción desencadena un nuevo despliegue en Render.
    \item \textbf{Optimización y Compilación:} Se realizaron optimizaciones de compilación para mejorar el rendimiento y reducir el tiempo de carga de la aplicación.
    \item \textbf{Pruebas y Verificación:} Se verificó que la aplicación web estuviera funcionando correctamente y se realizaron pruebas de rendimiento.
\end{itemize}