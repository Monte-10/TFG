\section{Requisitos Funcionales}

\subsection{Subsistema de Servidor y Comunicación con la Base de Datos}
\begin{itemize}
    \item \textbf{RF1} Será posible crear, modificar y eliminar modelos en el servidor que se almacenarán en la base de datos PostgreSQL.
    \item \textbf{RF2} Será posible crear y eliminar cualquier tabla en la base de datos PostgreSQL.
    \item \textbf{RF3} Será posible consultar, crear, editar y eliminar cualquier registro de las diferentes tablas en la base de datos mediante llamadas a través de la API REST.
\end{itemize}

\subsection{Subsistema de Gestión de Usuarios}
\begin{itemize}
    \item \textbf{RF4} Un administrador podrá registrar, modificar y eliminar cualquier usuario.
    \item \textbf{RF5} Un administrador podrá registrar, modificar y eliminar a otros administradores.
    \item \textbf{RF6} Un usuario/administrador podrá acceder a un panel de administración de Django.
    \item \textbf{RF7} Un usuario podrá ver su información personal en un panel de usuario.
    \item \textbf{RF8} Un administrador podrá ver la información de cualquier usuario en su panel de administración.
\end{itemize}

\subsection{Subsistema de Gestión de Entrenamientos y Nutrición}
\begin{itemize}
    \item \textbf{RF9} Un usuario podrá solicitar planes de entrenamiento y nutrición personalizados.
    \item \textbf{RF10} Un entrenador/nutricionista podrá crear, asignar y modificar planes de entrenamiento y nutrición.
    \item \textbf{RF11} Un usuario podrá visualizar y seguir los planes asignados por su entrenador/nutricionista.
    \item \textbf{RF12} Un entrenador/nutricionista podrá hacer seguimiento y ajustes en los planes de los usuarios.
\end{itemize}

\newpage

\section{Requisitos No Funcionales}

\subsection{Generales}
\begin{itemize}
    \item \textbf{RNF1} La aplicación tendrá una versión de desarrollo en la red local y otra de producción accesible desde cualquier navegador y dispositivo.
\end{itemize}

\subsection{Errores}
\begin{itemize}
    \item \textbf{RNF2} En caso de error en cualquier transacción con la base de datos, se devolverá un error específico que será manejado de manera consistente.
    \item \textbf{RNF3} La API gestionará y transmitirá los errores de transacciones de la base de datos al cliente.
    \item \textbf{RNF4} La lógica del front-end manejará estos errores y los mostrará apropiadamente al usuario.
\end{itemize}

\subsection{Seguridad}
\begin{itemize}
    \item \textbf{RNF5} Se implementará un sistema seguro basado en JSON Web Tokens, donde se asignará un token con cierta información a cada usuario.
    \item \textbf{RNF6} Se garantizará la autenticación basada en tokens para las operaciones de la base de datos solicitadas por la API.
    \item \textbf{RNF7} Las variables susceptibles se almacenarán de forma segura en variables de entorno.
    \item \textbf{RNF8} Las contraseñas de los usuarios se cifrarán mediante hashing con SHA256 antes de ser almacenadas en la base de datos.
    \item \textbf{RNF9} Las contraseñas cifradas nunca se devolverán en ninguna consulta.
    \item \textbf{RNF10} Ningún error debe conducir a la caída del sistema.
\end{itemize}

\subsection{Interfaz y Usabilidad}
\begin{itemize}
    \item \textbf{RNF11} La aplicación tendrá un alto grado de usabilidad en dispositivos más pequeños como móviles o tabletas.
    \item \textbf{RNF12} Todas las vistas y componentes se adaptarán al tamaño del dispositivo que los reproduzca.
    \item \textbf{RNF13} Todas las vistas y componentes seguirán un código de color y forma coherente.
    \item \textbf{RNF14} Se utilizarán iconos e imágenes de uso libre junto con creaciones propias para proporcionar un mejor diseño.
    \item \textbf{RNF15} Para generar una notificación de alarma, se abrirá un panel de confirmación basado en texto para evitar "clics erróneos".
    \item \textbf{RNF16} Todas las acciones críticas como eliminaciones abrirán un panel de confirmación para evitar "clics erróneos".
    \item \textbf{RNF17} Se mostrarán mensajes de error en texto cuando no se encuentre una página o el acceso esté prohibido/restringido.
\end{itemize}

\subsection{Soporte y Documentación}
\begin{itemize}
    \item \textbf{RNF18} La facilidad de acceso debe ser máxima simplemente ingresando la URL de la aplicación.
    \item \textbf{RNF19} No se requerirá instalación en la versión de ordenador (excepto para casos de prueba y simulación).
\end{itemize}

\subsection{Regulatorio y Legal}
\begin{itemize}
    \item \textbf{RNF20} Todos los recursos externos utilizados serán de uso libre y estarán debidamente acreditados en la documentación del proyecto.
\end{itemize}

\section{Incrementos\label{CAP:INCREMENTOS}}

\subsubsection{Back-end}
\begin{itemize}
    \item \textbf{IN1.1} Crear una base de datos PostgreSQL y configurar el servidor HTTP local con Django.
    \item \textbf{IN1.2} Definir modelos en Django para gestionar el módulo de deporte.
    \item \textbf{IN1.3} Definir modelos en Django para gestionar el módulo de nutrición.
    \item \textbf{IN1.4} Definir modelos en Django para gestionar los usuarios.
    \item \textbf{IN1.5} Definir vistas para controlar las operaciones de los modelos.
    \item \textbf{IN1.6} Crear controladores para manejar los inicios de sesión y registros de usuarios.
    \item \textbf{IN1.7} Crear APIs RESTful con Django REST framework para manejar las llamadas de los clientes.
    \item \textbf{IN1.8} Implementar la lógica de negocio y middleware necesarios para las operaciones del servidor.
    \item \textbf{IN1.9} Desplegar todo en Render.
\end{itemize}

\subsubsection{Front-end}
\begin{itemize}
    \item \textbf{IN2.1} Crear una aplicación básica con una barra de navegación y definir las rutas en React Router.
    \item \textbf{IN2.2} Desarrollar la sección de inicio de sesión.
    \item \textbf{IN2.3} Desarrollar el componente de inicio.
    \item \textbf{IN2.4} Desarrollar la sección de dietas.
    \item \textbf{IN2.5} Desarrollar la sección de entrenamientos.
    \item \textbf{IN2.6} Desarrollar la logica de autenticación y manejo de sesiones.
    \item \textbf{IN2.7} Desarrollar estetica y diseño de la aplicación.
    \item \textbf{IN2.8} Desarrollar sistema de filtraje y busqueda de dietas y entrenamientos.
    \item \textbf{IN2.9} Desplegar todo en Render.
\end{itemize}

\subsubsection{Aplicación Móvil}
\begin{itemize}
    \item \textbf{IN3.1} Configurar el entorno de desarrollo con React Native CLI.
    \item \textbf{IN3.2} Crear pantallas básicas para la navegación principal.
    \item \textbf{IN3.3} Implementar la autenticación y el manejo de sesiones.
    \item \textbf{IN3.4} Desarrollar la pantalla de inicio con acceso a las funcionalidades principales.
    \item \textbf{IN3.5} Implementar la funcionalidad de seguimiento de entrenamientos en tiempo real.
    \item \textbf{IN3.6} Desarrollar la funcionalidad de registro y seguimiento de dietas diarias.
    \item \textbf{IN3.7} Probar la aplicación en dispositivos iOS y Android.
    \item \textbf{IN3.8} Exportar la aplicación a una APK.
\end{itemize}