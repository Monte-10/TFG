En esta sección se presentan los códigos implementados en el proyecto. Se incluirán los fragmentos de código más relevantes.

\section{Back-End}
\subsection{Configuración de las Rutas y urls}
\begin{figure}[Ejemplo Código Urls]{FIG:urls}{Ejemplo de rutas en Django}
    \begin{verbatim}
    from django.urls import path, include
    from rest_framework.routers import DefaultRouter
    
    router = DefaultRouter()
    router.register(r'foods', views.FoodViewSet)
    router.register(r'assigned_options', views.AssignedOptionViewSet)
    
    urlpatterns = [
        path('', include(router.urls)),
        path('assign_option/', views.assignOption, name='assign_option'),
    ]
    \end{verbatim}
    \end{figure}


\subsection{Vistas y Controladores}
\begin{figure}[Ejemplo Código Vistas]{FIG:views}{Ejemplo de vistas en Django}
    \begin{verbatim}
    class TodayDailyDietView(APIView):
        def get(self, request, *args, **kwargs):
            today = timezone.now().date()
            user = request.user
            
            # Filtra las Dietas por el usuario logueado
            diets = Diet.objects.filter(user=user)
            
            # Encuentra las DailyDiet dentro del rango de fechas de las Dietas filtradas que coincidan con 'hoy'
            today_diets = DailyDiet.objects.filter(diet__in=diets, date=today)
            
            # Serializa y devuelve los resultados
            serializer = DailyDietSerializer(today_diets, many=True)
            return Response(serializer.data)

    @api_view(['POST'])
    def assignOption(request):
        if not request.user.is_trainer:
            return Response({"error": "Solo los entrenadores pueden crear asignaciones."}, status=status.HTTP_403_FORBIDDEN)
        
        user_id = request.data.get('userId')
        option_id = request.data.get('optionId')
    
        user = get_object_or_404(CustomUser, id=user_id)
        option = get_object_or_404(Option, id=option_id)
    
        assignment = AssignedOption.objects.create(
            user=user,
            option=option,
            start_date=timezone.now()
        )
    
        return Response({
            "message": "Option assigned successfully",
            "assignedOptionId": assignment.id,  # ID de la asignación
            "optionId": option.id,
            "start_date": assignment.start_date.strftime("%Y-%m-%d")
        }, status=status.HTTP_201_CREATED)
    \end{verbatim}
    \end{figure}

\newpage

\subsection{Serializadores}
\begin{figure}[Ejemplo Código Serializadores]{FIG:serializers}{Ejemplo de serializadores en Django}
    \begin{verbatim}
    class FoodSerializer(serializers.ModelSerializer):
        class Meta:
            model = Food
            fields = '__all__'

    class OptionSerializer(serializers.ModelSerializer):
        week_option_one = serializers.PrimaryKeyRelatedField(queryset=WeekOption.objects.all())
        week_option_two = serializers.PrimaryKeyRelatedField(queryset=WeekOption.objects.all())
        week_option_three = serializers.PrimaryKeyRelatedField(queryset=WeekOption.objects.all())

        class Meta:
            model = Option
            fields = ['id', 'name', 'week_option_one', 'week_option_two', 'week_option_three']
            read_only_fields = ['trainer']

        def create(self, validated_data):
            user = self.context['request'].user
            if user.is_trainer:
                trainer = user.trainer
            else:
                raise serializers.ValidationError("El usuario no está autorizado para crear opciones.")

            # Crea la instancia de Option incluyendo el trainer obtenido del contexto
            option = Option.objects.create(**validated_data, trainer=trainer)
            return option
            
    \end{verbatim}
    \end{figure}

\newpage

\subsection{Autenticación con JWT}
\begin{figure}[Vista Autenticación]{FIG:authview}{Vista de autenticación con JWT en Django}
    \begin{verbatim}
    class LoginView(APIView):
        authentication_classes = []  # No authentication required
        permission_classes = []  # No permission required
    
        def post(self, request, *args, **kwargs):
            username = request.data.get('username')
            password = request.data.get('password')
            user = authenticate(username=username, password=password)
            if user is not None:
                token, created = Token.objects.get_or_create(user=user)
                return Response(
                    {'token': token.key, 'userId': user.id}, 
                    status=status.HTTP_200_OK)
            else:
                return Response(
                    {'detail': 'Invalid Credentials'}, 
                    status=status.HTTP_401_UNAUTHORIZED)
    \end{verbatim}
    \end{figure}

\newpage

\subsection{Ejemplo de Modelo en Django}
\begin{figure}[Modelo de Entrenamiento]{FIG:trainingmodel}{Modelo de Entrenamiento en Django}
    \begin{verbatim}
class Option(models.Model):
    trainer = models.ForeignKey('user.Trainer',
     on_delete=models.CASCADE)
    name = models.CharField(max_length=100)
    week_option_one = models.ForeignKey(
        WeekOption, 
        related_name='week_option_one', on_delete=models.CASCADE)
    week_option_two = models.ForeignKey(
        WeekOption, related_name='week_option_two',
         on_delete=models.CASCADE)
    week_option_three = models.ForeignKey(
        WeekOption, related_name='week_option_three',
         on_delete=models.CASCADE)
    
    def __str__(self):
        return self.name

class Training(models.Model):
    trainer = models.ForeignKey(
        'user.Trainer', on_delete=models.CASCADE,
         related_name='trainer')
    name = models.CharField(max_length=255)
    exercises = models.ManyToManyField(Exercise, through='TrainingExercise')
    date = models.DateField()
    user = models.ForeignKey(
        User, on_delete=models.CASCADE, related_name='user')

    def __str__(self):
        return self.name
    \end{verbatim}
    \end{figure}

\newpage

\section{Front-End}

\subsection{Uso de Tokens en el Front-End}
\begin{figure}[Uso de Tokens]{FIG:usetokens}{Uso de tokens en el Front-end}
\begin{verbatim}
function Login({ onLoginSuccess }) {
    const [username, setUsername] = useState('');
    const [password, setPassword] = useState('');
    const [error, setError] = useState('');
    const navigate = useNavigate();
    const apiUrl = process.env.REACT_APP_API_URL;

    const handleSubmit = async (event) => {
        event.preventDefault();
        setError('');

        try {
            const response = await fetch(`
                ${apiUrl}/user/frontlogin/`, {
                method: 'POST',
                headers: {
                    'Content-Type': 'application/json',
                    'Authorization': `Token ${
                        localStorage.getItem('authToken')}`
                },
                body: JSON.stringify({ username, password }),
            });

            if (!response.ok) {
                const errorData = await response.json();
                throw new Error(
                    errorData.detail || 'Falló inicio de sesión');
            }

            const { token, userId } = await response.json();
            localStorage.setItem('authToken', token);
            localStorage.setItem('userId', userId);
            onLoginSuccess(token, userId);
            navigate('/dashboard');
        } catch (error) {
            setError(error.message);
        }
    };...}
\end{verbatim}
\end{figure}

\newpage

\subsection{Conexión con el Backend desde la Aplicación Web}
\begin{figure}[Ejemplo Servicio React]{FIG:reactservice}{Ejemplo de servicio en React para obtener planes de entrenamiento}
    \begin{verbatim}
        const apiUrl = process.env.REACT_APP_API_URL;

        export const getTrainingPlans = async () => {
            const response = await fetch(`${apiUrl}/sport/trainings/` {
                headers: {
                  'Authorization': `Token ${
                    localStorage.getItem('token')}`
                }
            });
            const data = await response.json();
            return data;
        };
    \end{verbatim}
    \end{figure}

\newpage

\subsection{Inicio Creación de una Dieta}

\begin{figure}[Creación de Dieta]{FIG:creatediet}{Creación de una dieta en React}
    \begin{verbatim}
function CreateDiet() {
  ...
  useEffect(() => {
    fetch(`${apiUrl}/user/regularusers/`, {
      headers: {
        'Authorization': `Token ${localStorage.getItem('authToken')}`
      }
    })
      .then(response => response.json())
      .then(data => {
        setUsers(data);
        if (data.length > 0) {
          setSelectedUser(data[0].id.toString());
        }
      });
  }, [apiUrl]);

  const handleSubmit = (event) => {
    event.preventDefault();
    const dietData = {
      name,
      user: selectedUser,
      start_date: startDate,
      end_date: endDate,
    };
    console.log("Sending diet data", dietData);
    fetch(`${apiUrl}/nutrition/diet/`, {
        method: 'POST',
        headers: {
            'Content-Type': 'application/json',
            'Authorization': `Token ${
                localStorage.getItem('authToken')}`
        },
        body: JSON.stringify(dietData),
    })
    ...
    \end{verbatim}
    \end{figure}

\newpage

\subsection{Return de una Dieta}
\begin{figure}[Return de Dieta]{FIG:returndiet}{Return de una dieta en React}
    \begin{verbatim}
return (
<div className="container mt-5">
    <h2 className="mb-4">Crear Nueva Dieta</h2>
    <form onSubmit={handleSubmit}>
    <div className="mb-3">
        <label htmlFor="name" 
        className="form-label">Nombre de la Dieta:</label>
        <input type="text" 
        className="form-control" id="name" value={name} 
        onChange={e => setName(e.target.value)} required />
    </div>
    <div className="mb-3">
        <label htmlFor="userSelect" 
        className="form-label">Usuario:</label>
        <select className="form-select" id="userSelect" 
        value={selectedUser} onChange={
            e => setSelectedUser(e.target.value)}
         required>
        <option value="">Seleccione un usuario</option>
        {users.map(user => (
            <option key={user.id} 
            value={user.id}>{user.username}</option>
        ))}
        </select>
    </div>
    <div className="mb-3">
        <label htmlFor="startDate" 
        className="form-label">Fecha de Inicio:</label>
        <input type="date" className="form-control" id="startDate" 
        value={startDate} onChange={e => setStartDate(e.target.value)}
         required />
    </div>
    ...
    <button type="submit" className="btn btn-primary">
    Crear Dieta</button>
    </form>
</div>);}
export default CreateDiet;
    \end{verbatim}
    \end{figure}


\newpage
\section{Aplicación Móvil}

\subsection{Conexión con el Backend desde Aplicación Móvil}
\begin{figure}[Ejemplo Servicio React Native]{FIG:reactnativeservice}{Ejemplo de servicio en React Native para obtener planes de entrenamiento}
    \begin{verbatim}
        import { API_URL } from '../config';
        export const fetchTrainingPlans = async () => {
            try {
                const response = await fetch(
                    `${API_URL}/training-plans/`);
                if (!response.ok) throw new Error(
                    'Network response was not ok');
                const data = await response.json();
                return data;
            } catch (error) {
                console.error('Error fetching training plans:', error);
                return [];
            }
        };
    \end{verbatim}
    \end{figure}

\newpage

\subsection{Configuración de la Navegación}
\begin{figure}[Configuración de Navegación]{FIG:navconfig}{Configuración de navegación en React Navigation}
    \begin{verbatim}
    const AppNavigator = () => {
        return (
            <Stack.Navigator initialRouteName="LoginScreen">
                <Stack.Screen name="HomeScreen" component={HomeScreen} />
                <Stack.Screen name="LoginScreen" component={LoginScreen} />
                <Stack.Screen name="TrainingDetailsScreen" 
                component={TrainingDetailsScreen} />
        </Stack.Navigator>
        );
    }
    export default AppNavigator;
    \end{verbatim}
    \end{figure}

\newpage

\subsection{DetailsScreen}

\begin{figure}[Pantalla de Detalles]{FIG:detailscreen}{Pantalla de detalles en React Native}
    \begin{verbatim}
const DetailsScreen = ({ route }) => {
const { date, dietEvents, trainingEvents } = route.params;
return (
    ...
    <Text style={styles.subtitle}>Entrenamientos:</Text>
    {trainingEvents && trainingEvents.length > 0 ? (
        trainingEvents.map((training, index) => (
        <View key={index} style={styles.section}>
            <Text style={styles.sectionTitle}>{training.name}</Text>
            {training.exercises_details.map((detail, idx) => (
            <Text key={idx} style={styles.itemText}>
                {detail.exercise.name} - 
                Series: {detail.sets}, ...
                {detail.weight && `, Peso: ${detail.weight}kg`}
                {detail.time && `, Tiempo: ${detail.time}min`}
            </Text>))}</View>))
    ) : (
        <Text style={styles.noDataText}>
        No hay entrenamientos para este día.</Text>
    )}
    <Text style={styles.subtitle}>Detalles de la Dieta Diaria:</Text>
    {dietEvents && dietEvents.length > 0 ? (
        ...
            <Text style={styles.sectionTitle}>Nutrientes:</Text>
            ...
            <Text style={styles.sectionTitle}>Comidas:</Text>
            {diet.mealsDetails && diet.mealsDetails.length > 0 ? (
            diet.mealsDetails.map((meal, mealIndex) => (
                <Text key={mealIndex} style={styles.itemText}>
                {meal.name}</Text>
            ))
            ) : (
            <Text style={styles.noDataText}>No hay ...</Text>
            )}
        </View>
    ...
    \end{verbatim}
    \end{figure}