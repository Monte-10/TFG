En esta sección se presentan los códigos implementados en el proyecto. Se incluirán los fragmentos de código más relevantes.

\section{Back-End}
\subsection{Configuración de las Rutas y urls}
\begin{figure}[Ejemplo Código Urls]{FIG:urls}{Ejemplo de rutas en Django}
    \begin{verbatim}
    from django.urls import path, include
    from rest_framework.routers import DefaultRouter
    
    router = DefaultRouter()
    router.register(r'foods', views.FoodViewSet)
    router.register(r'assigned_options', views.AssignedOptionViewSet)
    
    urlpatterns = [
        path('', include(router.urls)),
        path('assign_option/', views.assignOption, name='assign_option'),
    ]
    \end{verbatim}
    \end{figure}


\subsection{Vistas y Controladores}
\begin{figure}[Ejemplo Código Vistas]{FIG:views}{Ejemplo de vistas en Django}
    \begin{verbatim}
    class TodayDailyDietView(APIView):
        def get(self, request, *args, **kwargs):
            today = timezone.now().date()
            user = request.user
            
            # Filtra las Dietas por el usuario logueado
            diets = Diet.objects.filter(user=user)
            
            # Encuentra las DailyDiet dentro del rango de fechas de las Dietas filtradas que coincidan con 'hoy'
            today_diets = DailyDiet.objects.filter(diet__in=diets, date=today)
            
            # Serializa y devuelve los resultados
            serializer = DailyDietSerializer(today_diets, many=True)
            return Response(serializer.data)

    @api_view(['POST'])
    def assignOption(request):
        if not request.user.is_trainer:
            return Response({"error": "Solo los entrenadores pueden crear asignaciones."}, status=status.HTTP_403_FORBIDDEN)
        
        user_id = request.data.get('userId')
        option_id = request.data.get('optionId')
    
        user = get_object_or_404(CustomUser, id=user_id)
        option = get_object_or_404(Option, id=option_id)
    
        assignment = AssignedOption.objects.create(
            user=user,
            option=option,
            start_date=timezone.now()
        )
    
        return Response({
            "message": "Option assigned successfully",
            "assignedOptionId": assignment.id,  # ID de la asignación
            "optionId": option.id,
            "start_date": assignment.start_date.strftime("%Y-%m-%d")
        }, status=status.HTTP_201_CREATED)
    \end{verbatim}
    \end{figure}

\newpage

\subsection{Serializadores}
\begin{figure}[Ejemplo Código Serializadores]{FIG:serializers}{Ejemplo de serializadores en Django}
    \begin{verbatim}
    class FoodSerializer(serializers.ModelSerializer):
        class Meta:
            model = Food
            fields = '__all__'

    class OptionSerializer(serializers.ModelSerializer):
        week_option_one = serializers.PrimaryKeyRelatedField(queryset=WeekOption.objects.all())
        week_option_two = serializers.PrimaryKeyRelatedField(queryset=WeekOption.objects.all())
        week_option_three = serializers.PrimaryKeyRelatedField(queryset=WeekOption.objects.all())

        class Meta:
            model = Option
            fields = ['id', 'name', 'week_option_one', 'week_option_two', 'week_option_three']
            read_only_fields = ['trainer']

        def create(self, validated_data):
            user = self.context['request'].user
            if user.is_trainer:
                trainer = user.trainer
            else:
                raise serializers.ValidationError("El usuario no está autorizado para crear opciones.")

            # Crea la instancia de Option incluyendo el trainer obtenido del contexto
            option = Option.objects.create(**validated_data, trainer=trainer)
            return option
            
    \end{verbatim}
    \end{figure}

\newpage

\subsection{Autenticación con JWT}
\begin{figure}[Vista Autenticación]{FIG:authview}{Vista de autenticación con JWT en Django}
    \begin{verbatim}
    class LoginView(APIView):
        authentication_classes = []  # No authentication required
        permission_classes = []  # No permission required
    
        def post(self, request, *args, **kwargs):
            username = request.data.get('username')
            password = request.data.get('password')
            user = authenticate(username=username, password=password)
            if user is not None:
                token, created = Token.objects.get_or_create(user=user)
                return Response(
                    {'token': token.key, 'userId': user.id}, 
                    status=status.HTTP_200_OK)
            else:
                return Response(
                    {'detail': 'Invalid Credentials'}, 
                    status=status.HTTP_401_UNAUTHORIZED)
    \end{verbatim}
    \end{figure}

\newpage

\subsection{Ejemplo de Modelo en Django}
\begin{figure}[Modelo de Entrenamiento]{FIG:trainingmodel}{Modelo de Entrenamiento en Django}
    \begin{verbatim}
class Option(models.Model):
    trainer = models.ForeignKey('user.Trainer',
     on_delete=models.CASCADE)
    name = models.CharField(max_length=100)
    week_option_one = models.ForeignKey(
        WeekOption, 
        related_name='week_option_one', on_delete=models.CASCADE)
    week_option_two = models.ForeignKey(
        WeekOption, related_name='week_option_two',
         on_delete=models.CASCADE)
    week_option_three = models.ForeignKey(
        WeekOption, related_name='week_option_three',
         on_delete=models.CASCADE)
    
    def __str__(self):
        return self.name

class Training(models.Model):
    trainer = models.ForeignKey(
        'user.Trainer', on_delete=models.CASCADE,
         related_name='trainer')
    name = models.CharField(max_length=255)
    exercises = models.ManyToManyField(Exercise, through='TrainingExercise')
    date = models.DateField()
    user = models.ForeignKey(
        User, on_delete=models.CASCADE, related_name='user')

    def __str__(self):
        return self.name
    \end{verbatim}
    \end{figure}

\newpage

\section{Front-End}

\subsection{Uso de Tokens en el Front-End}
\begin{figure}[Uso de Tokens]{FIG:usetokens}{Uso de tokens en el Front-end}
\begin{verbatim}
function Login({ onLoginSuccess }) {
    const [username, setUsername] = useState('');
    const [password, setPassword] = useState('');
    const [error, setError] = useState('');
    const navigate = useNavigate();
    const apiUrl = process.env.REACT_APP_API_URL;

    const handleSubmit = async (event) => {
        event.preventDefault();
        setError('');

        try {
            const response = await fetch(`
                ${apiUrl}/user/frontlogin/`, {
                method: 'POST',
                headers: {
                    'Content-Type': 'application/json',
                    'Authorization': `Token ${
                        localStorage.getItem('authToken')}`
                },
                body: JSON.stringify({ username, password }),
            });

            if (!response.ok) {
                const errorData = await response.json();
                throw new Error(
                    errorData.detail || 'Falló inicio de sesión');
            }

            const { token, userId } = await response.json();
            localStorage.setItem('authToken', token);
            localStorage.setItem('userId', userId);
            onLoginSuccess(token, userId);
            navigate('/dashboard');
        } catch (error) {
            setError(error.message);
        }
    };...}
\end{verbatim}
\end{figure}

\newpage

\subsection{Conexión con el Backend desde la Aplicación Web}
\begin{figure}[Ejemplo Servicio React]{FIG:reactservice}{Ejemplo de servicio en React para obtener planes de entrenamiento}
    \begin{verbatim}
        const apiUrl = process.env.REACT_APP_API_URL;

        export const getTrainingPlans = async () => {
            const response = await fetch(`${apiUrl}/sport/trainings/` {
                headers: {
                  'Authorization': `Token ${
                    localStorage.getItem('token')}`
                }
            });
            const data = await response.json();
            return data;
        };
    \end{verbatim}
    \end{figure}

\newpage

\subsection{Inicio Creación de una Dieta}

\begin{figure}[Creación de Dieta]{FIG:creatediet}{Creación de una dieta en React}
    \begin{verbatim}
function CreateDiet() {
  ...
  useEffect(() => {
    fetch(`${apiUrl}/user/regularusers/`, {
      headers: {
        'Authorization': `Token ${localStorage.getItem('authToken')}`
      }
    })
      .then(response => response.json())
      .then(data => {
        setUsers(data);
        if (data.length > 0) {
          setSelectedUser(data[0].id.toString());
        }
      });
  }, [apiUrl]);

  const handleSubmit = (event) => {
    event.preventDefault();
    const dietData = {
      name,
      user: selectedUser,
      start_date: startDate,
      end_date: endDate,
    };
    console.log("Sending diet data", dietData);
    fetch(`${apiUrl}/nutrition/diet/`, {
        method: 'POST',
        headers: {
            'Content-Type': 'application/json',
            'Authorization': `Token ${
                localStorage.getItem('authToken')}`
        },
        body: JSON.stringify(dietData),
    })
    ...
    \end{verbatim}
    \end{figure}

\newpage

\subsection{Return de una Dieta}
\begin{figure}[Return de Dieta]{FIG:returndiet}{Return de una dieta en React}
    \begin{verbatim}
return (
<div className="container mt-5">
    <h2 className="mb-4">Crear Nueva Dieta</h2>
    <form onSubmit={handleSubmit}>
    <div className="mb-3">
        <label htmlFor="name" 
        className="form-label">Nombre de la Dieta:</label>
        <input type="text" 
        className="form-control" id="name" value={name} 
        onChange={e => setName(e.target.value)} required />
    </div>
    <div className="mb-3">
        <label htmlFor="userSelect" 
        className="form-label">Usuario:</label>
        <select className="form-select" id="userSelect" 
        value={selectedUser} onChange={
            e => setSelectedUser(e.target.value)}
         required>
        <option value="">Seleccione un usuario</option>
        {users.map(user => (
            <option key={user.id} 
            value={user.id}>{user.username}</option>
        ))}
        </select>
    </div>
    <div className="mb-3">
        <label htmlFor="startDate" 
        className="form-label">Fecha de Inicio:</label>
        <input type="date" className="form-control" id="startDate" 
        value={startDate} onChange={e => setStartDate(e.target.value)}
         required />
    </div>
    ...
    <button type="submit" className="btn btn-primary">
    Crear Dieta</button>
    </form>
</div>);}
export default CreateDiet;
    \end{verbatim}
    \end{figure}


\newpage
\section{Aplicación Móvil}

\subsection{Conexión con el Backend desde Aplicación Móvil}
\begin{figure}[Ejemplo Servicio React Native]{FIG:reactnativeservice}{Ejemplo de servicio en React Native para obtener planes de entrenamiento}
    \begin{verbatim}
        import { API_URL } from '../config';
        export const fetchTrainingPlans = async () => {
            try {
                const response = await fetch(
                    `${API_URL}/training-plans/`);
                if (!response.ok) throw new Error(
                    'Network response was not ok');
                const data = await response.json();
                return data;
            } catch (error) {
                console.error('Error fetching training plans:', error);
                return [];
            }
        };
    \end{verbatim}
    \end{figure}

\newpage

\subsection{Configuración de la Navegación}
\begin{figure}[Configuración de Navegación]{FIG:navconfig}{Configuración de navegación en React Navigation}
    \begin{verbatim}
    const AppNavigator = () => {
        return (
            <Stack.Navigator initialRouteName="LoginScreen">
                <Stack.Screen name="HomeScreen" component={HomeScreen} />
                <Stack.Screen name="LoginScreen" component={LoginScreen} />
                <Stack.Screen name="TrainingDetailsScreen" 
                component={TrainingDetailsScreen} />
        </Stack.Navigator>
        );
    }
    export default AppNavigator;
    \end{verbatim}
    \end{figure}

\newpage

\subsection{DetailsScreen}

\begin{figure}[Pantalla de Detalles]{FIG:detailscreen}{Pantalla de detalles en React Native}
    \begin{verbatim}
const DetailsScreen = ({ route }) => {
const { date, dietEvents, trainingEvents } = route.params;
return (
    ...
    <Text style={styles.subtitle}>Entrenamientos:</Text>
    {trainingEvents && trainingEvents.length > 0 ? (
        trainingEvents.map((training, index) => (
        <View key={index} style={styles.section}>
            <Text style={styles.sectionTitle}>{training.name}</Text>
            {training.exercises_details.map((detail, idx) => (
            <Text key={idx} style={styles.itemText}>
                {detail.exercise.name} - 
                Series: {detail.sets}, ...
                {detail.weight && `, Peso: ${detail.weight}kg`}
                {detail.time && `, Tiempo: ${detail.time}min`}
            </Text>))}</View>))
    ) : (
        <Text style={styles.noDataText}>
        No hay entrenamientos para este día.</Text>
    )}
    <Text style={styles.subtitle}>Detalles de la Dieta Diaria:</Text>
    {dietEvents && dietEvents.length > 0 ? (
        ...
            <Text style={styles.sectionTitle}>Nutrientes:</Text>
            ...
            <Text style={styles.sectionTitle}>Comidas:</Text>
            {diet.mealsDetails && diet.mealsDetails.length > 0 ? (
            diet.mealsDetails.map((meal, mealIndex) => (
                <Text key={mealIndex} style={styles.itemText}>
                {meal.name}</Text>
            ))
            ) : (
            <Text style={styles.noDataText}>No hay ...</Text>
            )}
        </View>
    ...
    \end{verbatim}
    \end{figure}

\newpage
\section{Pruebas del Back-End\label{SEC:PRUEBASBACKEND}}

\subsection{Configuracion de testsettings.py}
\begin{figure}[Configuración de Pruebas]{FIG:testsettings}{Configuración de pruebas en Django}
    \begin{verbatim}
from .settings import *
# Configuración de la base de datos para pruebas
DATABASES = {
    'default': {
        'ENGINE': 'django.db.backends.sqlite3',
        'NAME': ':memory:',
        'ATOMIC_REQUESTS': True,
    }
}
# Otros ajustes específicos para pruebas
SECRET_KEY = 'test-secret-key'
DEBUG = True
# Usar el backend de contraseñas de hashing rápido
PASSWORD_HASHERS = [
    'django.contrib.auth.hashers.MD5PasswordHasher',
]
# Desactivar middleware y aplicaciones en pruebas
MIDDLEWARE = [mw for mw in MIDDLEWARE if mw not in 
['debug_toolbar.middleware.DebugToolbarMiddleware']]
INSTALLED_APPS = [app for 
app in INSTALLED_APPS if app != 'debug_toolbar']

# Configuración de REST framework para pruebas
REST_FRAMEWORK['DEFAULT_AUTHENTICATION_CLASSES'] = (
    'rest_framework.authentication.SessionAuthentication',
    'rest_framework.authentication.BasicAuthentication',
)
# Configuración de CORS para pruebas (si es necesario)
CORS_ALLOW_ALL_ORIGINS = True
# Configuración de las rutas de archivos estáticos
STATIC_URL = '/static/'
MEDIA_URL = '/media/'
STATIC_ROOT = os.path.join(BASE_DIR, 'static_test')
MEDIA_ROOT = os.path.join(BASE_DIR, 'media_test')
# Configuración de la caché para pruebas
CACHES = {
    'default': {
    'BACKEND': 'django.core.cache.backends.locmem.LocMemCache',}}
    \end{verbatim}
    \end{figure}

\newpage

\subsection{Configuración de Pruebas de Usuario}
\begin{figure}[Configuración de Pruebas]{FIG:usertestsconfig}{Configuración de pruebas en Django}
    \begin{verbatim}
def setUp(self):
    self.client = APIClient(enforce_csrf_checks=False)
    self.user_data = {
        'username': 'testuser',
        'password': 'TestPassword123',
        'email': 'testuser@example.com',
        'first_name': 'Test',
        'last_name': 'User',
    }
    self.trainer_data = {
        'username': 'testtrainer',
        'password': 'TestPassword123',
        'email': 'testtrainer@example.com',
        'first_name': 'Test',
        'last_name': 'Trainer',
        'trainer_type': 'trainer'
    }
    \end{verbatim}
    \end{figure}

\newpage
\subsection{Pruebas de Usuario}
En la figura \ref{FIG:usertests} se presentan algunas de las pruebas más relevantes de usuario en Django.
\begin{figure}[Pruebas de Usuario]{FIG:usertests}{Pruebas de usuario en Django}
    \begin{verbatim}
def test_create_regular_user(self):
    url = reverse('regularuser_signup')
    response = self.client.post(url, self.user_data, format='json')
    self.assertEqual(response.status_code, status.HTTP_201_CREATED)
    self.assertEqual(CustomUser.objects.count(), 1)
    self.assertEqual(CustomUser.objects.get().username, 'testuser')

def test_create_trainer(self):
    url = reverse('trainer_signup')
    response = self.client.post(url, self.trainer_data, format='json')
    self.assertEqual(response.status_code, status.HTTP_201_CREATED)
    self.assertEqual(CustomUser.objects.count(), 1)
    self.assertEqual(CustomUser.objects.get().username, 'testtrainer')

def test_login_regular_user(self):
    self.test_create_regular_user()
    url = reverse('frontlogin')
    response = self.client.post(url, self.user_data, format='json')
    self.assertEqual(response.status_code, status.HTTP_200_OK)
    self.assertIn('token', response.data)

def test_search_trainer(self):
    self.test_create_regular_user()
    self.test_create_trainer()
    url = reverse('frontlogin')
    response = self.client.post(url, self.user_data, format='json')
    token = response.data['token']
    self.client.credentials(HTTP_AUTHORIZATION='Token ' + token)
    search_url = reverse('search_trainer')
    response = self.client.get(search_url, {'trainer_type': 'trainer'})
    self.assertEqual(response.status_code, status.HTTP_200_OK)
    self.assertGreaterEqual(len(response.data), 1)

    \end{verbatim}
    \end{figure}

\newpage

\subsection{Configuración de Pruebas de Deporte}
\begin{figure}[Configuración de Pruebas]{FIG:sporttestsconfig}{Configuración de pruebas en Django}
    \begin{verbatim}
def setUp(self):
    self.trainer_data = {
        'username': 'testtrainer',
        'password': 'TestPassword123',
        'email': 'testtrainer@example.com',
        'first_name': 'Test',
        'last_name': 'Trainer',
    }
    self.trainer = Trainer.objects.create_user(**self.trainer_data)
    self.client.force_authenticate(user=self.trainer)

    self.exercise_data = {
        'name': 'Push Up',
        'description': 'Push up exercise',
        'type': 'FUERZA'
    }

    self.training_data = {
        'trainer': self.trainer,
        'name': 'Morning Workout',
        'date': date.today(),
    }

    self.user_data = {
        'username': 'testuser',
        'password': 'TestPassword123',
        'email': 'testuser@example.com',
        'first_name': 'Test',
        'last_name': 'User',
    }
    self.user = CustomUser.objects.create_user(**self.user_data)
    \end{verbatim}
    \end{figure}

\newpage
\subsection{Pruebas de Deporte}
En la figura \ref{FIG:sporttests} se presentan algunas de las pruebas más relevantes de deporte en Django.
\begin{figure}[Pruebas de Deporte]{FIG:sporttests}{Pruebas de deporte en Django}
    \begin{verbatim}
def test_create_exercise(self):
    url = reverse('exercise-list')
    response = self.client.post(url, self.exercise_data, format='json')
    self.assertEqual(response.status_code, status.HTTP_201_CREATED)
    self.assertEqual(Exercise.objects.count(), 1)
    self.assertEqual(Exercise.objects.get().name, 'Push Up')

def test_create_training_exercise(self):
    exercise = Exercise.objects.create(**self.exercise_data)
    training = Training.objects.create(trainer=self.trainer,
     name='Morning Workout', date=date.today(), user=self.user)
    training_exercise_data = {
        'training': training.id,
        'exercise': exercise.id,
        'repetitions': 10,
        'sets': 3,
        'weight': 50,
    }
    url = reverse('trainingexercise-list')
    response = self.client.post(url, training_exercise_data,
     format='json')
    self.assertEqual(response.status_code, status.HTTP_201_CREATED)
    self.assertEqual(TrainingExercise.objects.count(), 1)
    self.assertEqual(TrainingExercise.objects.get().repetitions, 10)

def test_get_today_trainings(self):
    training = Training.objects.create(trainer=self.trainer,
     name='Morning Workout', date=date.today(), user=self.user)
    url = reverse('today-trainings')
    response = self.client.get(url, {'user': self.user.id})
    self.assertEqual(response.status_code, status.HTTP_200_OK)
    self.assertEqual(len(response.data), 1)
    self.assertEqual(response.data[0]['name'], 'Morning Workout')
    \end{verbatim}
    \end{figure}

\newpage

\subsection{Configuración de Pruebas de Nutrición}
\begin{figure}[Configuración de Pruebas]{FIG:nutritiontestsconfig}{Configuración de pruebas en Django}
    \begin{verbatim}
def setUp(self):
    self.trainer_data = {
        'username': 'testtrainer',
        'password': 'TestPassword123',
        'email': 'testtrainer@example.com',
        'first_name': 'Test',
        'last_name': 'Trainer',
    }
    self.trainer = Trainer.objects.create_user(**self.trainer_data)
    self.client.force_authenticate(user=self.trainer)

    self.food_data = {
        'name': 'Apple',
        'unit_weight': 100,
        'calories': 52,
        'protein': 0.26,
        'carbohydrates': 14,
        'sugar': 10,
        'fiber': 2.4,
        'fat': 0.17,
        'saturated_fat': 0.03,
    }

    self.ingredient_data = {
        'name': 'Apple Slice',
        'food': Food.objects.create(**self.food_data).id,
        'quantity': 150
    }

    self.user_data = {
        'username': 'testuser',
        'password': 'TestPassword123',
        'email': 'testuser@example.com',
        'first_name': 'Test',
        'last_name': 'User',
    }
    self.user = CustomUser.objects.create_user(**self.user_data)
    \end{verbatim}
    \end{figure}

\newpage
\subsection{Pruebas de Nutrición}
En la figura \ref{FIG:nutritiontests} se presentan algunas de las pruebas más relevantes de nutrición en Django.
\begin{figure}[Pruebas de Nutrición]{FIG:nutritiontests}{Pruebas de nutrición en Django}
    \begin{verbatim}
def test_create_food(self):
    url = reverse('food-list')
    response = self.client.post(url, self.food_data, format='json')
    self.assertEqual(response.status_code, status.HTTP_201_CREATED)
    self.assertEqual(Food.objects.count(), 1)
    self.assertEqual(Food.objects.get().name, 'Apple')

def test_create_dish(self):
    ingredient = Ingredient.objects.create(**self.ingredient_data)
    dish_data = {
        'user': self.user.id,
        'name': 'Apple Salad',
        'ingredients': [ingredient.id]
    }
    url = reverse('dish-list')
    response = self.client.post(url, dish_data, format='json')
    self.assertEqual(response.status_code, status.HTTP_201_CREATED)
    self.assertEqual(Dish.objects.count(), 1)
    self.assertEqual(Dish.objects.get().name, 'Apple Salad')

def test_get_today_dailydiets(self):
    diet = Diet.objects.create(user=self.user, name='Weight Loss', 
    start_date=timezone.now().date(), 
    end_date=timezone.now().date() + timedelta(days=7))
    daily_diet = DailyDiet.objects.create(diet=diet,
     date=timezone.now().date())
    daily_diet.meals.add(Meal.objects.create(user=self.user,
     name='Breakfast'))
    url = reverse('today-dailydiets')
    self.client.force_authenticate(user=self.user)
    response = self.client.get(url)
    self.assertEqual(response.status_code, status.HTTP_200_OK)
    self.assertEqual(len(response.data), 1)
    self.assertEqual(response.data[0]['diet'], diet.id)
    \end{verbatim}
    \end{figure}

\newpage

\section{Pruebas del Front-End\label{SEC:PRUEBASFRONTEND}}
En esta sección se presentan algunas de las pruebas realizadas en el front-end de la aplicación web.

\subsection{Pruebas de la Aplicación Web}
\subsubsection{Prueba de HomePage}
\begin{figure}[Prueba de HomePage.js]{FIG:homepagejs}{Prueba de HomePage.js en React}
    \begin{verbatim}
import React from 'react';
import { render } from '@testing-library/react';
import HomePage from './HomePage';

test('renders HomePage component', () => {
    const { getByText } = render(<HomePage />);
    const linkElement = getByText(/Welcome to HomePage/i);
    expect(linkElement).toBeInTheDocument();
});
    \end{verbatim}
    \end{figure}

\newpage

\subsection{Pruebas de Login}

\begin{figure}[Prueba de Login]{FIG:login}{Prueba de Login en React}
    \begin{verbatim}
import React from 'react';
import { render, fireEvent } from '@testing-library/react';
import Login from './Login';

test('renders Login component', () => {
    const { getByLabelText, getByText } = render(<Login />);
    const usernameInput = getByLabelText(/username/i);
    const passwordInput = getByLabelText(/password/i);
    const loginButton = getByText(/login/i);

    expect(usernameInput).toBeInTheDocument();
    expect(passwordInput).toBeInTheDocument();
    expect(loginButton).toBeInTheDocument();
});

test('allows the user to log in', () => {
    const { getByLabelText, getByText } = render(<Login />);
    const usernameInput = getByLabelText(/username/i);
    const passwordInput = getByLabelText(/password/i);
    const loginButton = getByText(/login/i);

    fireEvent.change(usernameInput, { target: { value: 'testuser' } });
    fireEvent.change(passwordInput, { target: { value: 'password' } });
    fireEvent.click(loginButton);
        });
    \end{verbatim}
    \end{figure}

\newpage

\subsection{Pruebas de Profile}
\begin{figure}[Prueba de Profile.js]{FIG:profilejs}{Prueba de Profile.js en React}
    \begin{verbatim}
import React from 'react';
import { render } from '@testing-library/react';
import Profile from './Profile';

test('renders Profile component', () => {
    const { getByText } = render(<Profile />);
    const headingElement = getByText(/Profile/i);
    expect(headingElement).toBeInTheDocument();
});
    \end{verbatim}
    \end{figure}

\newpage

\subsection{Pruebas de la Aplicación Móvil\label{SEC:PRUEBASMOVIL}}
En esta sección se presentan algunas de las pruebas realizadas en la aplicación móvil.

\subsubsection{Prueba de LoginScreen}

\begin{figure}[Prueba de LoginScreen]{FIG:loginscreen}{Prueba de LoginScreen en React Native}
    \begin{verbatim}
import React from 'react';
import { render, screen, fireEvent } from '@testing-library/react';
import LoginScreen from './LoginScreen';
import AuthContext from '../AuthContext';

const mockLogin = jest.fn();

describe('LoginScreen', () => {
    test('renders LoginScreen and performs login', () => {
    render(
        <AuthContext.Provider value={{ login: mockLogin }}>
        <LoginScreen />
        </AuthContext.Provider>
    );

    fireEvent.change(screen.getByPlaceholderText('Username'), {
        target: { value: 'testuser' },
    });
    fireEvent.change(screen.getByPlaceholderText('Password'), {
        target: { value: 'password123' },
    });
    fireEvent.click(screen.getByText('Login'));

    expect(mockLogin).toHaveBeenCalledWith('testuser', 'password123');
    });
});

    \end{verbatim}
    \end{figure}

\newpage

\subsubsection{Prueba de HomeScreen}
\begin{figure}[Prueba de HomeScreen]{FIG:homescreen}{Prueba de HomeScreen en React Native}
    \begin{verbatim}
import React from 'react';
import { render, screen } from '@testing-library/react';
import HomeScreen from './HomeScreen';

describe('HomeScreen', () => {
    test('renders HomeScreen with welcome message', () => {
    render(<HomeScreen />);
    expect(screen.getByText('Welcome to the Home Screen')).toBeInTheDocument();
    });
});
    \end{verbatim}
    \end{figure}

\newpage