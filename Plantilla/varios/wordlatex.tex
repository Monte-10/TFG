\section{Acceso a FitFuelBalance}

Para acceder a FitfuelBalance en su versión de producción, simplemente hay que acceder al link \url{https://fitfuelbalance.onrender.com} y registrarse (El Back-End se encuentra en \url(https://tfg-691w.onrender.com), pero no es necesario entrar para el uso de la aplicación). Una vez registrado, se puede acceder a la aplicación web como usuario o entrenador, ya que ambos roles están abiertos a cualquier usuario. También se proporcionan usuarios para acceder a la aplicación en la siguiente tabla:

\begin{table}[Usuarios para acceder a FitFuelBalance]{TAB:USUARIOS}{Usuarios para acceder a FitFuelBalance}
  \begin{tabular}{|p{3cm}|p{5cm}|p{5cm}|}
    \hline
    \textbf{Rol} & \textbf{Username} & \textbf{Password} \\
    \hline
    Entrenador & trainer & Tfg123@ \\
    \hline
    Usuario & user & Tfg123@ \\
    \hline
  \end{tabular}
\end{table}

La aplicación funciona en cualquier tipo de navegador, pero se recomienda no abrir varias sesiones con usuarios distintos en un mismo navegador para evitar errores.
\section{Acceso al código fuente}

El código fuente del Front-End, Back-End y Aplicación Móvil de FitFuelBalance se encuentra en los siguientes repositorios de GitHub:

\begin{itemize}
    \item \textbf{Front-End:} \url{https://github.com/Monte-10/TFG/tree/main/fitfuel-app}
    \item \textbf{Back-End:} \url{https://github.com/Monte-10/TFG/tree/main/fitfuelbalance}
    \item \textbf{Aplicación Móvil:} \url{https://github.com/Monte-10/TFG/tree/main/fitfuelmobile}
\end{itemize}

\section{Guía de Instalación}

Esta guía describe los pasos necesarios para ejecutar el proyecto FitFuelBalance en local para desarrollo y pruebas.

\subsection{Prerrequisitos}

Se debe tener \texttt{git}, \texttt{python3}, \texttt{pip}, \texttt{node.js}, y \texttt{npm} instalados en la máquina que se vaya a usar. Los siguientes comandos pueden ser utilizados para instalar estos prerrequisitos en una máquina con sistema operativo basado en Debian/Ubuntu:

\PythonCode[PRERREQUISITOS]{Instalación de Prerrequisitos.}{Instalación de git, python3, pip, node.js, y npm.}{instalacion/ins_prerrequisitos.py}{2}{22}{1}

\subsection{Descargar el Repositorio}

Clona el repositorio desde GitHub:

\PythonCode[INSTALACION]{Descargar Repositorio.}{Clonación del repositorio desde GitHub.}{instalacion/instalacion.py}{2}{3}{1}

\subsection{Backend (Django)}

Para configurar y ejecutar el backend de Django, sigue estos pasos:

\PythonCode[INSTALACION]{Configuración y Ejecución del Backend.}{Configuración y ejecución del backend de Django.}{instalacion/instalacion.py}{5}{17}{1}

\begin{itemize}
    \item Accede a la carpeta del backend.
    \item Instala las dependencias necesarias con pip.
    \item Ejecuta el servidor utilizando make. Esto realizará las migraciones y ejecutará el servidor.
    \item Alternativamente, puedes realizar las migraciones y ejecutar el servidor manualmente con los comandos de Django.
    \item Accede a la aplicación en tu navegador en \url{http://127.0.0.1:8000}.
\end{itemize}

\subsection{Frontend (React)}

Para configurar y ejecutar el frontend de React, sigue estos pasos:

\PythonCode[INSTALACION]{Configuración y Ejecución del Frontend.}{Configuración y ejecución del frontend de React.}{instalacion/instalacion.py}{19}{29}{1}

\begin{itemize}
    \item Accede a la carpeta del frontend.
    \item Usa la versión correcta de Node.js con nvm.
    \item Instala las dependencias necesarias con npm.
    \item Ejecuta la aplicación del frontend con npm start.
    \item Accede a la aplicación en tu navegador en \url{http://localhost:3000}.
\end{itemize}

\subsection{Frontend (React Native)}

Para configurar y ejecutar el frontend móvil de React Native, sigue estos pasos:

\PythonCode[INSTALACION]{Configuración y Ejecución del Frontend Móvil.}{Configuración y ejecución del frontend móvil de React Native.}{instalacion/instalacion.py}{31}{44}{1}

\begin{itemize}
    \item Accede a la carpeta del frontend móvil.
    \item Instala las dependencias necesarias con npm.
    \item Ejecuta la aplicación del frontend móvil con npm start.
    \item Verás un mensaje en la terminal indicando que el servidor de desarrollo está listo, con opciones para ejecutar en iOS, Android, abrir el menú de desarrollo, o recargar la aplicación.
\end{itemize}
