\label{sec:conclusiones}

\section{Conclusiones}
El desarrollo de FitFuelBalance ha sido una gran experiencia que ha resultado en la craeción de una herramienta moderna y avanzada para la gestión de planes de nutrición y ejercicio. A través de la combinación de tecnologías modernas y un enfoque centrado en el usuario, se ha logrado proporcionar una herramienta que facilita el acceso a planes personalizados, mejorando así la salud y el bienestar de los usuarios. Entre los logros más destacados del proyecto se encuentran el desarrollo integral, que incluye la implementación de un back-end sólido utilizando Django y una base de datos PostgreSQL, permitiendo una gestión eficiente y segura de los datos; la interfaz de usuario intuitiva, lograda mediante el uso de React para el desarrollo del front-end y React Native para la aplicación móvil, resultando en interfaces de usuario intuitivas y responsivas; el despliegue eficiente, facilitado por la utilización de plataformas como Render y Neon para el despliegue continuo y la gestión de la infraestructura; y la personalización de planes, permitiendo a los entrenadores y nutricionistas crear y ajustar planes personalizados de manera efectiva, atendiendo a las necesidades específicas de cada usuario. Este proyecto demuestra cómo la tecnología puede ser utilizada para mejorar significativamente la accesibilidad y eficacia de los servicios de nutrición y entrenamiento, promoviendo estilos de vida más saludables.

\section{Trabajo Futuro}
Aunque se han alcanzado los objetivos iniciales, hay margen para la mejora y expansión. A continuación, se destacan algunas áreas clave para el trabajo futuro:

\subsection{Mejoras en el Front-End}
Se planea continuar mejorando la experiencia del usuario mediante la optimización de la interfaz y la inclusión de nuevas funcionalidades interactivas, así como mejorar la estética y el diseño de la aplicación web y móvil para hacerla más atractiva y moderna.

\subsection{Mejoras en la Aplicación Móvil}
Se busca asegurar compatibilidad con cualquier tipo de dispositivos móviles y optimizar el rendimiento para una experiencia de usuario fluida, además de integrar notificaciones push para mantener a los usuarios informados sobre sus planes y recordatorios de actividades.

\subsection{Nuevas Funcionalidades}
Se propone desarrollar un sistema de recomendaciones personalizadas utilizando inteligencia artificial (IA), que pueda sugerir ajustes en los planes de nutrición y ejercicio basados en el progreso y las preferencias del usuario. También se busca implementar funcionalidades de seguimiento en tiempo real que permitan a los usuarios y entrenadores monitorizar el progreso de las actividades en directo.

\subsection{Integración de Inteligencia Artificial}
La mejora más importante y que más impulsaría la aplicación es la integración de IA, que podría incluir la generación automática de planes utilizando algoritmos de IA adaptados a las características individuales de los usuarios como peso, altura, alergias y objetivos específicos; la adaptación de planes existentes por parte de los entrenadores utilizando IA para adaptar planes a nuevos clientes con características diferentes; y el desarrollo de un asistente virtual que pueda interactuar con los usuarios, responder preguntas comunes y proporcionar soporte en tiempo real.
