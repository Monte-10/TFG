\label{sec:conclusiones}

\section{Conclusiones}
El desarrollo de FitFuelBalance ha sido un proceso desafiante y gratificante que ha culminado en la creación de una plataforma robusta y accesible para la gestión de planes de nutrición y ejercicio. A través de la combinación de tecnologías modernas y un enfoque centrado en el usuario, se ha logrado proporcionar una herramienta que facilita el acceso a planes personalizados, mejorando así la salud y el bienestar de los usuarios.

Entre los logros más destacados del proyecto se encuentran:
\begin{itemize}
    \item \textbf{Desarrollo Integral:} La implementación de un back-end sólido utilizando Django y una base de datos PostgreSQL ha permitido una gestión eficiente y segura de los datos.
    \item \textbf{Interfaz de Usuario Intuitiva:} El uso de React para el desarrollo del front-end y React Native para la aplicación móvil ha resultado en interfaces de usuario intuitivas y responsivas.
    \item \textbf{Despliegue Eficiente:} La utilización de plataformas como Render y Neon ha facilitado el despliegue continuo y la gestión de la infraestructura.
    \item \textbf{Personalización de Planes:} La plataforma permite a los entrenadores y nutricionistas crear y ajustar planes personalizados de manera efectiva, atendiendo a las necesidades específicas de cada usuario.
\end{itemize}

Este proyecto demuestra cómo la tecnología puede ser utilizada para mejorar significativamente la accesibilidad y eficacia de los servicios de nutrición y entrenamiento, promoviendo estilos de vida más saludables.

\section{Trabajo Futuro}
Aunque se han alcanzado unos objetivos iniciales, hay margen para la mejora y expansión al cual no se ha conseguido llegar. A continuación, se destacan algunas áreas clave para el trabajo futuro:

\subsection{Mejoras en el Front-End}
\begin{itemize}
    \item \textbf{Optimización de la Interfaz de Usuario:} Continuar mejorando la experiencia del usuario mediante la optimización de la interfaz y la inclusión de nuevas funcionalidades interactivas.
    \item \textbf{Mejoras en la estética:} Mejorar la estética y el diseño de la aplicación web y móvil para hacerla más atractiva y moderna.
\end{itemize}

\subsection{Mejoras en la Aplicación Móvil}
\begin{itemize}
    \item \textbf{Compatibilidad y Rendimiento:} Tener compatibilidad con cualquier tipo de dispositivos móviles y optimizar el rendimiento para asegurar una experiencia de usuario fluida.
    \item \textbf{Notificaciones Push:} Integrar notificaciones push para mantener a los usuarios informados sobre sus planes y recordatorios de actividades.
\end{itemize}

\subsection{Nuevas Funcionalidades}
\begin{itemize}
    \item \textbf{Sistema de Recomendaciones Personalizadas:} Desarrollar un sistema de recomendaciones personalizadas utilizando inteligencia artificial (IA), que pueda sugerir ajustes en los planes de nutrición y ejercicio basados en el progreso y las preferencias del usuario.
    \item \textbf{Seguimiento en Tiempo Real:} Implementar funcionalidades de seguimiento en tiempo real que permitan a los usuarios y entrenadores monitorizar el progreso de las actividades en directo.
\end{itemize}

\subsection{Integración de Inteligencia Artificial}
Una de las mejoras más prometedoras para el futuro de FitFuelBalance es la integración de IA. Esto podría incluir:
\begin{itemize}
    \item \textbf{Generación Automática de Planes:} Utilizar algoritmos de IA para generar planes de entrenamiento y nutrición automáticamente, adaptados a las características individuales de los usuarios como peso, altura, alergias, y objetivos específicos.
    \item \textbf{Adaptación de Planes Existentes:} Permitir que los entrenadores utilicen IA para adaptar planes existentes a nuevos clientes con características diferentes, asegurando así una personalización más rápida y precisa.
    \item \textbf{Asistente Virtual:} Desarrollar un asistente virtual que pueda interactuar con los usuarios, responder preguntas comunes y proporcionar soporte en tiempo real.
\end{itemize}